\documentclass[11pt, a4paper]{article}

\usepackage[brazil]{babel}
\usepackage[utf8]{inputenc}
\usepackage[T1]{fontenc}
\usepackage[pdftex]{hyperref}
\usepackage{graphicx}
\usepackage{amsmath}
\usepackage{indentfirst}
\usepackage{fancyhdr}


% Formatação
\topmargin -1.5cm
\oddsidemargin -0.04cm
\evensidemargin -0.04cm
\textwidth 16.59cm
\textheight 21.94cm 
%\pagestyle{empty}                     % Sem numero de paginas
\pagestyle{fancy}                         %cabecalhos e rodares
\fancyhead[RO,RE]{\today}
\fancyhead[LO,LE]{MAC0332 - SI para grupos de pesquisa\\
	Especificação dos Requerimentos}
\fancyfoot[LO,LE]{Confidential}
\fancyfoot[RO,RE] {\thepage}
\fancyfoot[CO,CE]{Grupo 3, 2011}


\parskip 7.2pt                        % Espaço entre paragrafos 7.2
%\renewcommand{\baselinestretch}{1.5} % Espaçamento entre linhas = 1.5
%\parindent 0pt

% Tirar hifenização
\hyphenpenalty = 5000
\tolerance = 1000
\sloppy

\title{MAC 0332\\
	Engenharia de Software\\
	SI para grupos de pesquisa\\
	System-Wide Requirements}
\date{\today}
\author{Grupo(nomes e NUSP?)}

\begin{document}

	\maketitle
	\newpage
	
	\section{Introdução}
		Esse documento descreve as qualidades do sistema como objetivos e 
		funcionalidades do mesmo que o cliente requer. Tambem descreve as 
		ferramentas que serão usadas assim como seu custo e viabilidade para 
		o projeto. 
	\section{Requerimentos Funcionais do Sistema}
		O programa devera ter um cadastro de \textit{usuarios} (membros) onde 
		\indent alguns tambem serão \textit{administradores}, um tipo de usuario 
		especial.\\
		\indent \textit{Adminstradores} são os unicos que podem cadastrar 
		grupos e subgrupos.\\
		\indent Um \textit{membro} do grupo cadastra uma publicação e pode 
		dizer a quais projetos aquela publicação se aplica.\\
		\indent O \textit{membro} tambem pode Fazer uploads de PDF's.
		
	\section{Qualidades do Sistema}
		Nessa seção sera descrito as funcionalidades do sistema, seu uso, 
		confiabilidade, performance e suporte.
		
		\subsection{Uso}
			O sistema deve ser de facil entendimento para que os 
			\textit{membros} e \textit{administradores} o usem sem maiores 
			dificuldades. Ou pelo menos seja de facil aprendizado.
		
		\subsection{Confiabilidade}
			%[Reliability includes the product and/or system's ability to keep 
			%running under stress and adverse conditions. Specify requirements 
			%for reliability acceptance levels, and how they will be measured 
			%and evaluated. Suggested topics are availability, frequency of 
			%severity of failures and recoverability.]
			Acho que ainda não da pra fazer isso.
		
		\subsection{Performance}
			%[The performance characteristics of the system should be outlined 
			%in this section. Examples are response time, throughput, capacity 
			%and startup or shutdown times.]
			Acho que ainda não da pra fazer isso
		
		\subsection{Suporte}
			%[This section indicates any requirements that will enhance the 
			%supportability or maintainability of the system being built, 
			%including adaptability and upgrading, compatibility, 
			%configurability, scalability and requirements regarding system 
			%installation, level of support and maintenance.]
			Acho que ainda não da pra fazer isso
		
	\section{Interfaces do Sistema}
		%[Interface Requirements are part of the + in the FURPS+ classification 
		%of supporting requirements. Define the interfaces that must be 
		%supported by the application. It should contain adequate specificity, 
		%protocols, ports and logical addresses, and so forth, so that the 
		%software can be developed and verified against the interface requirements.]
		Nessa seção definimos as interfaces que deverão ser suportadas pela 
		aplicação.
		
		\subsection{Interface do Usuario}
			%[Describe the user interfaces that are to be implemented by the 
			%software. The intention of this section is to state requirements 
			%relating to the interface. Interface design may overlap the 
			%requirements gathering process.]
			Descrição das interfaces do usuario implementadas por software.
			
			\subsubsection{Aparencia}
				%[Provide a description of the spirit of the interface. Your 
				%client may have given you particular demands such as style, 
				%colors to be used, and degree of interaction and so on. This 
				%section captures the requirements for the interface rather 
				%than the design for the interface.]
				Não nos foi dada nenhuma especificação sobre a aparencia, ainda.
				
			\subsubsection{Layout e Requerimentos de Navegação}
				O sistema deve dividido em parte publica, a que todos os 
				visitantes podem ver, e parte restrita na qual só os membros 
				terão acesso.\\
				\\
				\noindent \textbf{Parte publica deve ter:}
				\begin{itemize}
					\item lista das disciplinas oferecidas pelo grupo;
					\item lista dos projetos oferecidos pelo grupo, com 
					descrição, membros do projeto e publicações;
					\item lista das publicações do grupo, separados por tipo e 
					se disponivel seu PDF tambem deve ser mostrado;
					\item na lista de publicação deve ser possivel ordenar por 
					tipo, por ano ou por veículo;
					\item lista de  membros do grupo;
					\item lista das linhas de pesquisa organizadas 
					hierarquicamente.
				\end{itemize}
				
				\indent Alem disso um membro do grupo cadastra uma publicação e
				pode dizer a quais projetos aquela publicação se aplica.\\
				\indent Ao clicar em uma linha de pesquisa deve aparecer sua 
				descrição e tambem as publicações, os membros do grupo e os 
				projetos associados a ela.
							
			\subsubsection{Consistencia}

			\subsubsection{Requerimentos de personalização pelo usuario}

		\subsection{Interfaces para sistemos externos ou dispositivos}
			\subsubsection{Interfaces do Software}
			
			\subsubsection{Interfaces do Hardware}
			
			\subsubsection{Interfaces de Comunicação}
			
	\section{Componentes do Sistema}

\end{document}

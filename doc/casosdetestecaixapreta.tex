\documentclass[11pt, a4paper]{book}

\usepackage[brazil]{babel}
\usepackage[utf8]{inputenc}
\usepackage[T1]{fontenc}
\usepackage[pdftex]{hyperref}
\usepackage{graphicx}
\usepackage{amsmath}
\usepackage{indentfirst}
\usepackage{fancyhdr}
\usepackage{setspace}         


\usepackage{color}
\usepackage{pifont}
\usepackage{amsfonts}
\usepackage{amssymb}
\usepackage{amsmath}
\usepackage{setspace}                  
\usepackage[small,compact]{titlesec} 		

\usepackage[square,sort,nonamebreak]{natbib}
\usepackage[Lenny]{fncychap}					
\usepackage{url}
\usepackage{latexsym}
\usepackage{multicol}
\usepackage{multirow}

% Formatação
\topmargin -1.5cm
\oddsidemargin -0.04cm
\evensidemargin -0.04cm
\textwidth 16.59cm
\textheight 21.94cm 
%\pagestyle{empty}                     % Sem numero de paginas
\pagestyle{fancy}                         %cabecalhos e rodares
\fancyhead[RO,RE]{\today}
\fancyhead[LO,LE]{MAC0332 - SI para grupos de pesquisa\\
	Casos de Teste Caixa Preta para Análise Funcional}
\fancyfoot[LO,LE]{Confidencial}
\fancyfoot[RO,RE] {\thepage}
\fancyfoot[CO,CE]{Grupo 3, 2011}


\parskip 7.2pt                        % Espaço entre paragrafos 7.2
%\renewcommand{\baselinestretch}{1.5} % Espaçamento entre linhas = 1.5
%\parindent 0pt

% Tirar hifenização
\hyphenpenalty = 5000
\tolerance = 1000
\sloppy


% numeração para subsubsection
%errado: \renewcommand\thesubsubsection{\@arabic\c@section.\@arabic\c@subsection.\@arabic\c@subsubsection.}
\setcounter{secnumdepth}{3}
\setcounter{tocdepth}{4}


\begin{document}

% Capa
\thispagestyle{empty}
\begin{center}
    \vspace*{0.2cm}
    \textbf{\Large{Sistema de Informação para Grupos de Pesquisa}}\\
	
    \vspace*{1.2cm}
    \Large{Analista de Qualidade: Jorge J. G. Leandro}\\
    \Large{Grupo 3}
    
    \vskip 2cm
	\textsc{
	MAC0332 - 2011\\[-0.25cm] 
          	Engenharia de Software\\[-0.25cm] 	
	IME - USP\\[-0.25cm]
	}
    
    \vskip 1.5cm
    Casos de Teste Caixa Preta\\
    Prof: Marco Gerosa\\

	
    \vskip 0.5cm
 %   \normalsize{São Paulo, Julho de 2008}
   {\normalsize São Paulo, \today}
\end{center}


\chapter[Apresentação]{Apresentação}
\label{cap:apresentacao}	
	
	 	O presente documento descreve o cumprimento da tarefa \emph{Criar casos de teste}, mediante uma coleção de Casos de Teste Caixa Preta da Análise Funcional para o SIGP, de acordo com [Schach, 2007] e os gabaritos do \emph{Processo Unificado Aberto - OpenUP}.

\chapter[Casos de Teste]{Casos de Teste Caixa Preta - Análise Funcional}
\label{cap:casosdeteste}

	\section{Caso de Teste - SIGPCT001 }
	\begin{itemize}
	\item Descrição: Cadastrar uma \textbf{pessoa}.
	\item Pré-condições: Ser administrador. Ter uma \emph{View} com formulário para cadastro de pessoa. Banco de Dados criado.
	\item Pós-condições: O registro de mais uma pessoa no BD e apresentação na listagem subsequente.
	\item Dados necessários: Nome, CPF, Publicações, Grupos
	\end{itemize}

	\section{Caso de Teste - SIGPCT002 }
	\begin{itemize}
	\item Descrição: Ser administrador. Ter uma \emph{View} com formulário para cadastro de pessoa com entradas para especificar pessoa que é membro-usuário. Cadastrar um \textbf{membro-usuário}.
	\item Pré-condições: Banco de Dados criado.
	\item Pós-condições: O registro de mais um membro usuário no BD e na listagem subsequente.
	\item Dados necessários: Nome, CPF, Publicações, Grupos, Login, Senha, Avatar, Tipo
	\end{itemize}

	\section{Caso de Teste - SIGPCT003}
	\begin{itemize}
	\item Descrição: Cadastrar um \textbf{membro não-usuário}.
	\item Pré-condições: Ser administrador. Ter uma \emph{View} com formulário para cadastro de pessoa com entradas para especificar pessoa que é membro não-usuário. Banco de Dados criado.
	\item Pós-condições: O registro de mais um membro usuário no BD e na listagem subsequente.
	\item Dados necessários: Nome, CPF, Publicações, Grupos, Login, Senha, Avatar, Tipo
	\end{itemize}

	\section{Caso de Teste - SIGPCT004}
	\begin{itemize}
	\item Descrição: Cadastrar um \textbf{não-membro não-usuário}.
	\item Pré-condições: Ser administrador. Ter uma \emph{View} com formulário para cadastro de pessoa com entradas para especificar pessoa que é não-membro não-usuário. Banco de Dados criado.
	\item Pós-condições: O registro de mais um membro usuário no BD e na listagem subsequente.
	\item Dados necessários:  Nome, CPF, Publicações, Grupos, Login, Senha, Avatar, Tipo
	\end{itemize}

	\section{Caso de Teste - SIGPCT005}
	\begin{itemize}
	\item Descrição: Cadastrar um \textbf{grupo}.
	\item Pré-condições: Ser administrador. Ter uma \emph{View} com formulário para cadastro de grupo. Ter o Banco de Dados criado.
	\item Pós-condições: O registro de mais um grupo no BD e na listagem subsequente.
	\item Dados necessários: Nome, Lista de linhas de pesquisa.
	\end{itemize}

	\section{Caso de Teste - SIGPCT006}
	\begin{itemize}
	\item Descrição: Cadastrar um \textbf{subgrupo}.
	\item Pré-condições: Ser administrador. Ter uma \emph{View} com formulário para cadastro de subgrupo de um grupo.Ter o Banco de Dados criado.
	\item Pós-condições: O registro de mais um subgrupo no BD e na listagem subsequente.
	\item Dados necessários: Nome, Nome do grupo, Lista de linhas de pesquisa.
	\end{itemize}

	\section{Caso de Teste - SIGPCT007}
	\begin{itemize}
	\item Descrição: Cadastrar uma \textbf{publicação}.
	\item Pré-condições: Ser membro-usuário. Ter uma \emph{View} com formulário para cadastro de publicação.Ter o Banco de Dados criado.
	\item Pós-condições: O registro de mais uma publicação no BD e na listagem subsequente.
	\item Dados necessários: 
	\end{itemize}

	\section{Caso de Teste - SIGPCT008}
	\begin{itemize}
	\item Descrição: Relacionar uma \textbf{publicação} aos \textbf{projetos} correlatos.
	\item Pré-condições: Ser membro-usuário. Ter uma \emph{View} com campos para relacionar publicações a projetos.Ter o Banco de Dados criado.
	\item Pós-condições: O registro de mais uma publicação no BD e na listagem subsequente.
	\item Dados necessários: Título, veículo, Data, Grupo, Lista de Autores
	\end{itemize}

	\section{Caso de Teste - SIGPCT009}
	\begin{itemize}
	\item Descrição: Fazeer \emph{upload} de arquivo \emph{.pdf} de uma \textbf{publicação}.
	\item Pré-condições: Ser membro-usuário. Ter uma \emph{View} com entrada para caminho do arquivo. Banco de Dados criado.
	\item Pós-condições: O registro de mais uma disciplina no BD e na listagem subsequente.
	\item Dados necessários: Nome do grupo. Arquivo \emph{.pdf} da publicação. 
	\end{itemize}

	\section{Caso de Teste - SIGPCT010}
	\begin{itemize}
	\item Descrição: Cadastrar uma \textbf{linha de pesquisa}.
	\item Pré-condições: Ter uma \emph{View} com formulário para cadastrar linha de pesquisa. Banco de Dados criado.
	\item Pós-condições: O registro de mais uma linha de pesquisa no BD e na listagem subsequente.
	\item Dados necessários: Nome. Lista de projetos na linha de pesquisa.
	\end{itemize}

	\section{Caso de Teste - SIGPCT011}
	\begin{itemize}
	\item Descrição: Cadastrar um \textbf{projeto}.
	\item Pré-condições: Banco de Dados criado.
	\item Pós-condições: O registro de mais um projeto no BD e na listagem subsequente.
	\item Dados necessários: Descrição do projeto, Nome da Agência de Fomento financiadora, lista de grupos participantes, lista de pessoas participantes.
	\end{itemize}
    

\end{document}

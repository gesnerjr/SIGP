\documentclass[11pt, a4paper]{article}

\usepackage[brazil]{babel}
\usepackage[utf8]{inputenc}
\usepackage[T1]{fontenc}
\usepackage[pdftex]{hyperref}
\usepackage{graphicx}
\usepackage{amsmath}
\usepackage{indentfirst}
\usepackage{fancyhdr}


% Formatação
\topmargin -1.5cm
\oddsidemargin -0.04cm
\evensidemargin -0.04cm
\textwidth 16.59cm
\textheight 21.94cm 
%\pagestyle{empty}                     % Sem numero de paginas
\pagestyle{fancy}                         %cabecalhos e rodares
\fancyhead[RO,RE]{\today}
\fancyhead[LO,LE]{MAC0332 - SI para grupos de pesquisa\\
	Especificação dos Requerimentos}
\fancyfoot[LO,LE]{Confidential}
\fancyfoot[RO,RE] {\thepage}
\fancyfoot[CO,CE]{Grupo 3, 2011}


\parskip 7.2pt                        % Espaço entre paragrafos 7.2
%\renewcommand{\baselinestretch}{1.5} % Espaçamento entre linhas = 1.5
%\parindent 0pt

% Tirar hifenização
\hyphenpenalty = 5000
\tolerance = 1000
\sloppy

\title{MAC 0332\\
	Engenharia de Software\\
	SI para grupos de pesquisa\\
	System-Wide Requirements}
\date{\today}

\begin{document}

	\maketitle
	\newpage
	
	\section{Introdução}
		Esse documento descreve as qualidades do sistema como objetivos e 
		funcionalidades do mesmo que o cliente requer. Tambem descreve as 
		ferramentas que serão usadas assim como seu custo e viabilidade para 
		o projeto. 
	\section{Requesitos Funcionais do Sistema}
		%[Statement of system-wide functional requirements, not expressed as 
		% use cases. Examples include auditing, authentication, printing, 
		% reporting.]
		% =====================================================================
		% Usuários e administradoes, parte 1.
		O sistema deverá ter um cadastro de \textbf{usuarios}, que poderão ser
		\textbf{administradores} -- um tipo especial de usuario -- ou não. Usuários possuem
		\textit{login}, senha e avatar e podem comentar sobre \textbf{publicações}. Adminstradores
		são os únicos que podem cadastrar \textbf{grupos de pesquisa}, \textbf{linhas de pesquisa} e
		\textbf{contribuintes}.\\
		% Grupos de pesquisa.
		\indent Grupos de pesquisa podem ter \textbf{subgrupos de pesquisa} e/ou serem subgrupo de
		outro grupo de pesquisa. Cada grupo de pesquisa pode também ministrar \textbf{disciplinas}, 
		e de conduzir pelo menos uma linha de pesquisa. Uma disciplina é ministrada por exatamente
		um grupo de pesquisa. Só o administrador pode cadastrar novos grupos de pesquisa.\\
		% Linhas de pesquisa.
		\indent Linhas de pesquisa, assim como os grupos de pesquisa, possuem uma hierarquia
		recursiva própria, isso é, podem ter \textbf{sublinhas de pesquisa} ou serem sublinha de
		outra linha de pesquisa. Linhas de pesquisa podem conter diversos \textbf{projetos}. Só o
		administrador pode cadastrar novas linhas de pesquisa.\\
		% Contribuintes.
		\indent Contribuintes devem ter pelo menos uma \textbf{filiação} com algum grupo de pesquisa
		ou ser autor de alguma publicação. Só o administrador pode cadastrar novos contribuintes, e
		deverá no momento do cadastro satisfazer pelo menos uma das duas condições anteriores. Todo
		usuário, com excessão do administrador, deve corresponder a um contribuinte. Contribuintes
		podem também estudar em linhas de pesquisa e participar de projetos.\\
		\indent A filiação de um contribuinte a um grupo de pesquisa possui uma data de início e uma
		data de fim. Analogamente, o período de tempo que um contribuinte passa pesquisando em uma
		linha de pesquisa é delimitado por uma data de início e data de fim. Contribuintes podem
		estar filiados somente a grupos de pesquisa que não tenham subgrupos de pesquisa, e será
		subentendido que ele está filiado a todos os grupos acima deste grupo. No entanto, um
		contribuinte poderá pesquisar em qualquer linha de pesquisa, sendo igualmente subentendido
		que ele pesquisa em todas as linhas de pesquisa acima desta.\\
		\indent Projetos possuem uma descrição e um financiamento e podem produzir publicações.
		Qualquer contribuinte que esteja estudando em uma linha de pesquisa pode cadastrar um novo
		projeto nela.\\
		\indent Publicações possuem título, veículo, data de publicação e possivelmente um arquivo
		pdf associado a ela que contenha a publicação em si. Além disso, as publicações possuem
		autores, que devem ser contribuintes, como já mencionado. Qualquer usuário contribuinte pode
		cadastrar uma publicação e dizer a quais projetos ela se aplica, assim como fazer
		\textit{upload} do arquivo pdf para aquela publicação.
		
	\section{Qualidades do Sistema}
		%[Qualities represent the URPS in FURPS+ classification of 
		%supporting requirements.]
		Nessa seção são descritas as qualidades que o sistema deverá apresentar.
		
		\subsection{Uso}
			%[Describe requirements for qualities such as easy of use, 
			%easy of learning, usability standards and localization.]
			O sistema deve ser de facil entendimento para que os 
			\textit{contribuintes} e \textit{administradores} o usem sem maiores 
			dificuldades. Ou pelo menos seja de facil aprendizado.\\
			\indent Todo o conteúdo e todas as operações deverão ser acessível do
			\textit{WebSite}, contanto que o usuário tenha as devidas permissões de acessá-los.
		
		%\subsection{Confiabilidade}
			%[Reliability includes the product and/or system's ability to keep 
			%running under stress and adverse conditions. Specify requirements 
			%for reliability acceptance levels, and how they will be measured 
			%and evaluated. Suggested topics are availability, frequency of 
			%severity of failures and recoverability.]
			%Acho que ainda não da pra fazer isso.
		
		\subsection{Desempenho}
			%[The performance characteristics of the system should be outlined 
			%in this section. Examples are response time, throughput, capacity 
			%and startup or shutdown times.]
			Não há requesitos de desempenho. 
		
		\subsection{Suporte}
			%[This section indicates any requirements that will enhance the 
			%supportability or maintainability of the system being built, 
			%including adaptability and upgrading, compatibility, 
			%configurability, scalability and requirements regarding system 
			%installation, level of support and maintenance.]
			O sistema deve ser de facil manutenção e atualização.
		
	\section{Interface do Sistema com o Usuário}
		%[Interface Requirements are part of the + in the FURPS+ classification 
		%of supporting requirements. Define the interfaces that must be 
		%supported by the application. It should contain adequate specificity, 
		%protocols, ports and logical addresses, and so forth, so that the 
		%software can be developed and verified against the interface 
		%requirements.]
		Nessa seção definimos como deverá ser a interface do usuário do nosso sistema.
		%[Describe the user interfaces that are to be implemented by the 
		%software. The intention of this section is to state requirements 
		%relating to the interface. Interface design may overlap the 
		%requirements gathering process.]
		
		\subsection{Aparencia}
			%[Provide a description of the spirit of the interface. Your 
			%client may have given you particular demands such as style, 
			%colors to be used, and degree of interaction and so on. This 
			%section captures the requirements for the interface rather 
			%than the design for the interface.]
			Os html's devem ser bem estruturados possibilitando mudanças na
			aparencia quando necessario. Além disso, devem ser capazes de
			listar de maneira intuitiva informações com caráter hierárquico.\\
			\indent Não precisa ser nada muito exagerado, mas sim objetivo.
			
		\subsection{Layout e Requerimentos de Navegação}
			%[Capture requirements on major screen areas and how they 
			%should be grouped together.]
			O sistema deve ser dividido em uma parte pública, que todos os
			visitantes podem ver, e duas partes restritas: uma à qual todos
			os usuários terão acesso, e outra à qual apenas os administradores
			terão acesso.
			
			\noindent \textbf{A parte publica fornece:}
			\begin{itemize}
				\item Listagem dos grupo de pesquisa;
				\item Listagem das disciplinas oferecidas por um grupo;
				\item Listagem dos projetos oferecidos por um grupo, com
				descrição, membros do projeto e publicações;
				\item Listagem das publicações do grupo, separados por veículo
				e, se disponivel, seu PDF também deve ser mostrado;
				\item Nessa listagem de publicações deve ser possivel ordenar
				por título, por ano ou por veículo;
				\item Listagem dos contribuintes filiados a um grupo;
				\item Listagem das linhas de pesquisa organizadas
				hierarquicamente;
				\item Exibição da descrição de uma linha de pesquisam, assim
				como a listagem de suas publicações, contribuintes e projetos
				associados a ela;
				\item Exibição da descrição e financiamento de um projeto.
			\end{itemize}
			
			\noindent \textbf{A parte dos usuários contribuintes fornece:}
			\begin{itemize}
				\item Cadastro de um novo projeto, indicando a quais linhas de
				pesquisa ele se aplica;
				\item Cadastro de uma nova publicação, indicando a quais
				projetos ela se aplica e podendo fazer \textit{upload} do
				arquivo PDF que contenha a publicação em si;
				\item Registro de que está estudando determinada linha de
				pesquisa;
				\item Registro de que está participando de um determinado
				projeto.
			\end{itemize}
			
			\noindent \textbf{A parte dos administradores fornece:}
			\begin{itemize}
				\item Cadastro de um novo grupo de pesquisa;
				\item Cadastro de uma nova linha de pesquisa;
				\item Cadastro de um novo contribuinte;
				\item Cadastro de uma nova filiação de um contribuinte com
				algum grupo de pesquisa.
				\item Autenticação de um usuário como correspondente a um
				contribuinte.
			\end{itemize}
			
			Vale lembrar que tudo que o administrador é apenas um tipo especial
			de usuário, e portanto também tem acesso à interface dos usuários
			contribuintes.
						
		\subsection{Consistência}
			% [Consistency in the user interface enables users to predict 
			% what will happen. This section states requirements on the use 
			% of mechanisms to be employed in the user interface. This 
			% applies both within the system and with other systems and 
			% can be applied at different levels: navigation controls, 
			% screen areas sizes and shapes, placements for entering / 
			% presenting data, terminology.]
			Como uma boa organização dos html's essa parte pode ser 
			facilmente modificada, caso for necessario.\\
			\indent Basicamente, haverá três tipos de operações na interface do
			sistema: listagem, detalhamento e cadastros. A princípio, o usuário
			pode clicar em links que forneçam as listagens básicas.
			Selecionando um item de uma listagem, aparecerá o detalhamento
			dele, que pode envolver outras listagens. Em qualquer um desses
			dois casos, dependendo do nível de permissão do usuário, links
			possibilitandos cadastros poderão estar disponíveis. Por exemplo,
			no detalhamento de um projeto, um usuário contribuinte poderá
			escolher uma opção do tipo "Particar desse projeto".

		\subsection{Requerimentos de personalização pelo usuario}
			%[Requirements on content that should automatically displayed 
			% to users or available based on user attributes. Sometimes users
			% allowed to customize the content displayed or to personalize
			% displayed content.]
			\noindent Os requerimentos de personalização pelo usuário são:
			\begin{itemize}
				\item Suporte a mais linguagens, definidas pelo usuário;
				\item Avatar para usuarios;
				\item Comentários dos usuarios em publicações.
			\end{itemize}
				
    \section{Componentes do Sistema}
		\noindent O sistema deverá ter as seguintes componentes:
		\begin{itemize}
			\item Um banco de dados para os grupos de pesquisa;
			\item Um \textit{Web Site} através do qual o sistema interagirá com
			seus diversos usuários;
			\item Uma aplicação que gerenciará os requisitos dessa interface
			\textit{web}, fazendo as devidas consultas ao banco de dados e
			fornecendo corretamente as informações e operações solicitadas;
			\item E um servidor Tomcat, no qual ficarão hospedado as
			componentes acima.
		\end{itemize}
    

\end{document}

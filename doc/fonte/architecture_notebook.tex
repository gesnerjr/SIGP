\documentclass[11pt, a4paper]{article}

\usepackage[brazil]{babel}
\usepackage[utf8]{inputenc}
\usepackage[T1]{fontenc}
\usepackage[pdftex]{hyperref}
\usepackage{graphicx}
\usepackage{amsmath}
\usepackage{indentfirst}
\usepackage{fancyhdr}


% Formatação
\topmargin -1.5cm
\oddsidemargin -0.04cm
\evensidemargin -0.04cm
\textwidth 16.59cm
\textheight 21.94cm 
%\pagestyle{empty}                     % Sem numero de paginas
\pagestyle{fancy}                         %cabecalhos e rodares
\fancyhead[RO,RE]{\today}
\fancyhead[LO,LE]{MAC0332 - SI para grupos de pesquisa\\
	Especificação dos Requerimentos}
%\fancyfoot[LO,LE]{Confidential}
\fancyfoot[RO,RE] {\thepage}
\fancyfoot[CO,CE]{Grupo 3, 2011}


\parskip 7.2pt                        % Espaço entre paragrafos 7.2
%\renewcommand{\baselinestretch}{1.5} % Espaçamento entre linhas = 1.5
%\parindent 0pt

% Tirar hifenização
\hyphenpenalty = 5000
\tolerance = 1000
\sloppy

\title{MAC 0332\\
	Engenharia de Software\\
	SI para grupos de pesquisa\\
	Architecture Notebook}
\date{\today}

\begin{document}

	\maketitle
	\newpage
	
	\section{Propósito}
        Este documento descreve a filosofia, as decisões, as restrições, as
        justificativas, os elementos significativos e qualquer outro aspecto
        importante do sistema que dão forma ao seu \emph{design} e a sua implementação.

    %
    %[Always address Sections 2 through 6 of this template. Other sections are
    % recommended, depending on the amount of novel architecture, the amount of
    % expected maintenance, the skills of the development team, and the
    % importance of other architectural concerns.]
    %
    \section{Objetivos arquiteturais e filosofia}
        %
        %[Describe the philosophy of the architecture. Identify issues that will
        % drive the philosophy, such as: Will the system be driven by complex
        % deployment concerns, adapting to legacy systems, or performance
        % issues? Does it need to be robust for long-term maintenance?
        %
        % Formulate a set of goals that the architecture needs to meet in its
        % structure and behavior. Identify critical issues that must be
        % addressed by the architecture, such as: Are there hardware
        % dependencies that should be isolated from the rest of the system? Does
        % the system need to function efficiently under unusual conditions?]
        %
        A arquitetura desse projeto visa ao desenvolvimento de um sistema
        organizado, durável e prático, tanto para desenvolvedores, quanto para
        usuários.

        Por meio de ferramentas que abstraem implementações de recursos
        normalmente complexos (como banco de dados e interfaces \emph{Web}), podemos,
        de certa forma, limitar o escopo de desenvolvimento à parte da lógica do
        sistema.

        Mediante esta decisão arquitetural e de projeto, simplificamos o desenvolvimento e reduzimos em grande parte o volume de código necessário, o que facilitará sobremaneira sua documentação, verificação e manutenção.

        Logo, os principais objetivos arquiteturais serão garantir que:

        \begin{enumerate}
            \item Estas ferramentas sejam devidamente aplicadas no sistema, isto é, que elas estejam corretamente instaladas e instanciadas,
            e que nosso sistema as use e as faça comunicarem-se entre si de
            maneira bem sucedida, aproveitando ao máximo suas facilidades;

            \item A lógica do sistema esteja de acordo com os requerimentos do
            cliente, priorizando sua testabilidade e manutenabilidade, uma
            vez que a maioria das tarefas tecnicamente complexas estará sendo coberta
            pelas ferramentas auxiliares.
        \end{enumerate}

    \section{Suposições e dependências}
        %
        %[List the assumptions and dependencies that drive architectural
        % decisions. This could include sensitive or critical areas,
        % dependencies on legacy interfaces, the skill and experience of the
        % team, the availability of important resources, and so forth]
        %
        \subsection{Dependências tecnológicas do projeto:}
            \begin{itemize}
                \item Um servidor Tomcat, com a versão 6, para hospedar nosso
                sistema;
                
                \item Um banco de dados MySQL, para persistir os dados
                gerenciados pelo sistema;
                
                \item a JDK e o Eclipse (Enterprise Edition ou equivalente),
                para desenvolvimento do sistema usando Java para Web;
                
                \item Um navegador para visualizarmos a interface Web do 
                sistema;
                
                \item A biblioteca JDBC e seu plugin para banco de dados MySQL, para viabilizar a conexão da aplicação Java com o banco de dados;
                
                \item A implementação Hibernate da JPA, para abstrair o banco de
                dados por meio de um mapeamento objeto-relacional, facilitando
                o desenvolvimento do sistema em Java, que é uma linguagem
                orientada a objetos;
                
                \item O \emph{framework} VRaptor, que implementa o controle da arquitetura MVC, para controlar e comunicar facilmente a aplicação Java com a interface Web do sistema;
                
                \item As bibliotecas JUnit e Mockito para execução de testes automatizados do sistema;
				
				\item As bibliotecas JQuery e JQueryUI para interatividade nas
				páginas Web;
                
                \item O sistema Git para controle de versão e concorrência do código do projeto;
                
                \item Quaisquer outras dependências indiretas geradas pelas
                tecnologias citadas.
            \end{itemize}

        \subsection{Dependências de conhecimento e experiência da equipe:}
            \begin{itemize}
                \item Programação em linguagem Java usando a IDE Eclipse;
                
                \item Análise, Modelagem e Desenvolvimento com orientação a objetos;
                
                \item Produção de páginas Web em XHTML (com CSS);

	    \item Competência para desenvolvimento de páginas dinâmicas Web com a linguagem Java, mediante o uso de \emph{Expression Language} e JSTL (taglibs).
                
                \item Modelagem e projeto de bancos de dados;
                
                \item Competência para desenvolvimento com as bibliotecas Hibernate, VRaptor, JQuery, 
				JUnit e Mockito;
                
                \item Competência para desenvolvimento com o sistema de controle de versão e concorrência Git.
            \end{itemize}

        \subsection{Suposições:}
            \begin{itemize}
                \item Assumimos que as ferramentas, bibliotecas e demais
                tecnologias usadas no desenvolvimento do projeto são funcionais,
                isto é, agem como o esperado, segundo os registros de suas respectivas documentações, dispensando, por assim dizer, testes adicionais.
            \end{itemize}

    \section{Requisitos para realizar a arquitetura}
        %
        %[Insert a reference or link to the requirements that must be
        % implemented to realize the architecture.]
        %
        Para que se possa implementar a arquitetura descrita neste documento, a
        equipe de desenvolvimento deve ter os seguintes recursos devidamente
        instalados e funcionando:

        \begin{itemize}
            \item Ambiente de desenvolvimento Java usando \emph{Eclipse Enterprise
            Edition} (ou equivalente);
            
            \item Um servidor Tomcat devidamente configurado para rodar a partir do Eclipse;
            
            \item Um repositório remoto Git, onde serão mantidos o código fonte do
            projeto e os diversos artefatos importantes que forem produzidos ao
            longo do projeto (inclusive este);
            
            \item Um projeto em branco de VRaptor do site da Caelum importado no
            Eclipse. Este projeto servirá como gabarito base para o aplicativo Java que conterá a lógica do sistema. Este projeto deve ser adicionado ao
            repositório Git, preferencialmente por meio do \textit{plug-in} Git do
            Eclipse.
            Feito isto, todos os membros da equipe poderão sincronizar seus repositórios locais ao 
            repositório remoto, com o objetivo de unirem-se ao desenvolvimento do sistema (ou seja, só
            uma pessoa precisa realizar esta tarefa e, em seguida, todos poderão importar este
            projeto a partir do repositório remoto Git, usando o Eclipse);
            
            \item Os JAR's das bibliotecas JDBC, Hibernate (com \emph{annotations}),
            JUnit (versão 4) e Mockito, bem como suas respectivas dependências,
            incluídas na pasta WebContent/WEB-INF/lib do projeto Java.
            A biblioteca VRaptor já está inclusa no projeto em branco da Caelum.
            (Do mesmo modo que no item anterior, uma vez que alguém tenha
            conseguido fazer isto, basta usar o repositório remoto Git para que todos obtenham suas cópias locais do projeto,  igualmente configuradas.)
            %
            % Falta alguma coisa?
            %
        \end{itemize}

    \section{Decisões, restrições e justificativas}
        %
        %[List the decisions that have been made regarding architectural
        % approaches and the constraints being placed on the way that the
        % developers build the system. These will serve as guidelines for
        % defining architecturally significant parts of the system. Justify each
        % decision or constraint so that developers understand the importance of
        % building the system according to the context created by those
        % decisions and constraints. This may include a list of DOs and DON’Ts
        % to guide the developers in building the system.] 
        %
        \begin{itemize}
            \item O sistema será programado em Java, pois é uma linguagem
            orientada a objetos, guarnecida de diversas facilidades e recursos que
            auxiliam tanto no desenvolvimento, quanto na manutenção do sistema.
            Além disso, a profícua comunidade desta linguagem provê diversas ferramentas, as quais 
            facilitam a aplicação de Java na construção de sistemas de software
            pertinentes às exigências do mercado e dos círculos de pesquisa.
            
            \item O ambiente de desenvolvimento será o \emph{Eclipse Enterprise
            Edition} (ou equivalente), pois é equipado com todas as ferramentas
            básicas para desenvolvimento para Web, usando Java. Em particular, ele
            oferece a possibilidade de rodar um servidor Tomcat local no próprio
            Eclipse, o que possibilita que os desenvolvedores possam testar o
            sistema em suas próprias máquinas a qualquer momento, como se o
            estivessem rodando no servidor remoto.

            \item Para comunicar nossa aplicação Java com a interface Web do
            sistema, usaremos o \emph{framework} VRaptor. Como esta implementa o controlador da arquitetura 
MVC, ela simplifica drasticamente  a construção de controladores, além de gerenciar automaticamente as
            instâncias das diversas partes do sistema, criando-as e mantendo-as
            disponíveis conforme demandas solicitadas pelo cliente Web.

            \item Para comunicar nossa aplicação Java com o banco de dados,
            usaremos a biblioteca Hibernate, que, por sua vez, se conectará ao banco de dados por meio da biblioteca JDBC. Devido ao mapeamento objeto-relacional
            proporcionado pelo Hibernate (de acordo com a especificação JPA), prescindimos da 
complexidade da criação manual das
            \textit{queries} ao banco de dados, haja vista que esta tarefa será realizada  automaticamente. O emprego deste \emph{framework} permite que a programação orientada a objetos para banco de dados seja realizada de maneira quase que natural, propiciada pelo Hibernate que viabiliza a correspondência entre classes e relações.

            \item Para realizar os testes de unidade, usaremos as
            bibliotecas JUnit e Mockito. A JUnit é uma das ferramenta para testes de unidade em 
			Java largamente utilizada. Além disso, o Eclipse tem \textit{plug-ins} que auxiliam seu uso. 
A Mockito possibilita a simulação de objetos e respectivos comportamentos, possibilitando a realização de testes de unidade,  que de fato não dependam de outras unidades para serem testados.
			
			\item Para facilitar o desenvolvimento das páginas Web, usamos a 
			biblioteca JQuery, caracterizada pela facilidade de uso e por disponibilizar ferramentas
			poderosas, com suporte a diversos navegadores. Isto permite que os
			desenvolvedores concentrem-se no desenvolvimento de novas funcionalidades, em vez de 			diferenças entre navegadores.
			
			\item Para elementos de interface Web, a JQuery UI fornece diversos
			recursos avançados, por meio de comandos simples. É possível criar rapidamente calendários, telas com abas ou até mesmo formulários dinâmicos.
			
            %
            % \item [Decision or constraint and justification]
            % 
            % Wilson: "Adicionarei outras decisões conforme elas forem sendo
            % feitas. Por enquanto deixei apenas aquilo que já está decidido
            % pelas exigências do professor sobre o projeto."
            %
        \end{itemize}
    
    \section{Mecanismos arquiteturais}
        %
        %[List the architectural mechanisms and describe the current state of
        % each one. Initially, each mechanism may be only name and a brief
        % description. They will evolve until the mechanism is a collaboration
        % or pattern that can be directly applied to some aspect of the design.]
        %
        Esta é uma das partes que evoluirão ao longo do projeto, moldando-se de
        acordo com as necessidades que forem surgindo.

        \subsection{Controlador}
            Módulo responsável por controlar a parte lógica, sob a interface Web.

            Propósito: Fornecer as páginas web pedidas pelo usuário, assim
			como executar qualquer lógica de alguma página em particular, como
			por exemplo, o cadastro de um usuário.

            Funcionamento: Cada endereço \emph{URI} solicitado pelo usuário deve estar associado
			a algum método de algum controlador, assim chamado, passando
			os dados que o usuário enviou juntamente com a requisição. Este método irá
			processar o pedido, redirecionando o resultado para uma página web.

        \subsection{Modelo - \textit{DAO - Data Access Object}}
            Módulo responsável por armazenar e fornecer acesso a um certo tipo de dado do
            banco de dados.

            Propósito: Encapsular o Hibernate, integrando-se melhor ao sistema
			e facilitando testes e manutenção.

            Funcionamento: Por ser uma classe do nosso próprio sistema, é mais
			facilmente simulada (\emph{mocked}) e integrada ao VRaptor.
			Cada método é basicamente uma chamada direta ao Hibernate, sendo que algumas
			também realizam \emph{Transactions}.

        \subsection{Injeção de Dependência}
            Método utilizado para lidar com dependências de uma classe com
            outra. Sempre que uma classe precisar de um objeto de outra,
            deverá dar-se preferência ao uso de injeções de dependência.

            Propósito: reduzir o acoplamento a classes de outras bibliotecas, melhorar a legibilidade e a testabilidade do código.

            Funcionamento: Quando o VRaptor instanciar um controle, aquele verificará que o controle depende de um objeto de acesso a dados (DAO) e criará uma instância deste. Isto é possível, pois o objeto de acesso a dados fora devidamente anotado, para que o VRaptor tenha conhecimento de que, doravante, deverá gerenciar as instâncias deste objeto.

        \subsection{\textit{Factories}}
            Módulo que facilita a criação de objetos com configurações
            pré-determinadas. Compatível com injeção de dependências.

            Propósito: Fornecer objetos prontos para o uso.

            Funcionamento: Uma classe com diversos métodos, onde cada um
			devolve um dos objetos desejados já configurados.

        \subsection{Testes de unidade}
            Método utilizado para testar isolada e individualmente unidades do sistema.

            Propósito: Garantir que cada classe do projeto funcione individualmente.

            Funcionamento: Para cada classe do projeto (excluindo aquelas que 
			apenas encapsulam uma biblioteca externa), cria-se uma classe que implementa um teste de unidade,  utilizando o JUnit. Neste tipo de teste, as saídas dos métodos são comparadas a valores esperados.
        
        %
        % \subsection{Architectural Mechanism}
        %   [Describe the purpose, attributes, and function of the architectural
        %    mechanism.]
        %

    \section{Abstrações chave}
        %
        %[List and briefly describe the key abstractions of the system. This
        % should be a relatively short list of the critical concepts that define
        % the system. The key abstractions will usually translate to the initial
        % analysis classes and important patterns.]
        %
        \begin{description}
            \item[Interface Web:] conjunto de páginas \emph{web} que compõem o \emph{Website}
            do sistema e que podem ser visualizadas pelos usuários. Envia
            requisições para o sistema, conforme ações dos usuários e
            exibe, ao usuário, as informações que o sistema lhe forneceu em
            resposta.

            \item[Banco de dados:] conjunto de dados armazenados em um servidor, conforme um modelo relacional (usando MySQL). Os dados
            armazenados dizem respeito aos grupos de pesquisa, usuários,
            publicações e demais informações que foram passadas ao sistema de
            grupos de pesquisa.

            \item[Lógica do sistema:] aplicativo (Java) que gerencia a lógica do
            sistema, interagindo com os usuários através da Interface Web e
            persistindo os dados obtidos no Banco de Dados.
            %
            % Wilson: "Dá pra colocar mais algumas coisas mas começaria a ficar
            % repetitivo com a seção anterior. Coloco mesmo assim?"
            %
        \end{description}

    \section{Camadas da arquitetura}
        %
        %[Describe the architectural pattern that you will use or how the
        % architecture will be consistent and uniform. This could be a simple
        % reference to an existing or well-known architectural pattern, such as
        % the Layer framework, a reference to a high-level model of the
        % framework, or a description of how the major system components should
        % be put together.]
        %
        Há, no total, seis camadas na arquitetura do projeto. A primeira, e mais
        externa, é a interface Web, composta pelos diversos arquivos
        \texttt{*.jsp}, \texttt{*.css} e \texttt{*.js} que têm sido escritos, formando as 
		diversas páginas Web exibidas aos usuários quando do acesso ao sistema.
        A segunda camada é o controle da arquitetura MVC, iimplementada pelo VRaptor, que serve para ligar a primeira camada com a terceira, ou seja, com as classes de controle do sistema. Estas, por sua vez, comunicam-se 
        com a quarta camada, a saber, a lógica do sistema. É nesta, que o trabalho da equipe 
tem se concentrado durante o desenvolvimento do sistema.  A quinta camada é composta pelos \textit{Data Access Objects} (DAOs), que viabiliza o armazenamento de dados e permite que a quarta camada (lógica do sistema) se comunique  com a  sexta e última camada, que é o banco de dados. Este, por sua vez, é abstraído pelo  mapeamento objeto-relacional do Hibernate.
        %
        % Obs.1 Wilson: "assim que puder colocarei uma imagem na próxima seção
        % que ilustrará bem essas camadas"
        %
        % Obs.2 Wilson: "se for necessário posso detalhar melhor od que cada
        % camada será composta" (Samuel: "acho que não")
        %

    \section{Visões arquiteturais}
        %
        %[Describe the architectural views that you will use to describe the
        % software architecture. This illustrates the different perspectives
        % that you will make available to review and to document architectural
        % decisions.]
        %
        % --- Recommended views ---
        %
        % *Logical: Describes the structure and behavior of architecturally
        % significant portions of the system. This might include the package
        % structure, critical interfaces, important classes and subsystems, and
        % the relationships between these elements. It also includes physical
        % and logical views of persistent data, if persistence will be built
        % into the system. This is a documented subset of the design.
        %
        % *Operational: Describes the physical nodes of the system and the
        % processes, threads, and components that run on those physical nodes.
        % This view isn’t necessary if the system runs in a single process and
        % thread.
        %
        % *Use case: A list or diagram of the use cases that contain
        % architecturally significant requirements.
        %
        %
        \begin{itemize}
            \item \textbf{Visão lógica:}
                \\

            % \item \textbf{Visão das classes:} %TODO
            
            \item \textbf{Casos de uso com requisitos arquiteturalmente
            importantes:}
                \\
                Um relato completo dos casos de uso levantados pela equipe
                estão no documento ``Modelo de Casos de Uso'', que acompanha
                esta documentação.
                \\\\
                Alguns casos considerados \textit{particular ou exemplarmente}
                relevantes com relação à arquitetura do sistema -- por exemplo,
                por nos forçarem a garantir certas relações diretas ou
                indiretas entre módulos ou unidades do sistema, ou a realização
                de certas estruturas de interface com o usuário, são:
                \begin{itemize}
                    \item Administrador cadastra um grupo de pesquisa;
                    
                    \item Administrador cadastra um contribuinte e o filia a
                    um grupo;
                    
                    \item Administrador aponta um contribuinte como autor de
                    uma publicação;
                    
                    \item Contribuinte cadastra um projeto em uma linha de
                    pesquisa que estuda;
                    
                    \item Contribuinte cadastra uma publicação e cita quais
                    os projetos aos quais ela se aplica;

                    \item Usuário realiza login;
                    
                    \item Visitante lista disciplinas oferecidas por um grupo;
                    
                    \item Visitante lista grupos e linhas de pesquisa;
                    
                    \item Visitante lista projetos oferecidos por uma linha
                    pesquisa;
                    
                    \item Visitante lista projetos oferecidos por um grupo.

		\item Visitante lista Contribuintes de um projeto, Linha ou um Grupo

		\item Visitante lista Publicações de um projeto, Linha ou um Grupo
                    
                \end{itemize}
        \end{itemize}
\end{document}

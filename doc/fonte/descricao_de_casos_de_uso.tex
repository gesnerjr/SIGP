\documentclass[11pt, a4paper,oneside]{book}

\usepackage[brazil]{babel}
\usepackage[utf8]{inputenc}
\usepackage[T1]{fontenc}
%\usepackage[pdftex]{hyperref}
\usepackage[pdftex]{graphicx}  
\usepackage[square,sort,nonamebreak]{natbib}


\usepackage{amsmath}
\usepackage{indentfirst}
\usepackage{fancyhdr}
\usepackage{setspace}         
\usepackage[pagebackref,colorlinks=true,urlcolor=cyan,citecolor=red,linkcolor=blue]{hyperref}

\usepackage{color}
\usepackage{pifont}
\usepackage{amsfonts}
\usepackage{amssymb}
\usepackage{amsmath}
\usepackage{setspace}                  
\usepackage[small,compact]{titlesec} 		

%\usepackage[Lenny]{fncychap}					
\usepackage[Bjornstrup]{fncychap}
\usepackage{url}
\usepackage{latexsym}
\usepackage{multicol}
\usepackage{multirow}

% Formatação
\topmargin -1.5cm
\oddsidemargin -0.04cm
\evensidemargin -0.04cm
\textwidth 16.59cm
\textheight 21.94cm 
%\pagestyle{empty}                     % Sem numero de paginas
\pagestyle{fancy}                         %cabecalhos e rodares

\rhead{}

%\fancyhead[RO,RE]{\today}
%\fancyhead[LO,LE]{MAC0332 - SIGP - Sistema de Informação para Grupos de Pesquisa\\
%Descrições de Casos de Uso}

\fancyhead[LO,LE]{
\centering
\begin{tabular}[h]{|l|l|}
\hline
MAC0332 & SIGP - Sistema de Informação para Grupos de Pesquisa \\
\hline
Descrição de Casos de Uso & Data: \today \\
\hline 
\end{tabular}
}

%\pagestyle{myheadings}
%\markright{
%\centering
%\begin{tabular}[h]{|l|l|}
%\hline
%MAC0332 & SIGP - Sistema de Informação para Grupos de Pesquisa \\
%\hline
%Descrição de Casos de Uso & Data: \today \\
%\hline 
%\end{tabular}
%}

%\fancyfoot[LO,LE]{Confidencial}
\fancyfoot[RO,RE] {\thepage}
\fancyfoot[CO,CE]{Grupo 3, 2011}


\parskip 7.2pt                        % Espaço entre paragrafos 7.2
%\renewcommand{\baselinestretch}{1.5} % Espaçamento entre linhas = 1.5
%\parindent 0pt

% Tirar hifenização
\hyphenpenalty = 5000
\tolerance = 1000
\sloppy


% numeração para subsubsection
%errado: \renewcommand\thesubsubsection{\@arabic\c@section.\@arabic\c@subsection.\@arabic\c@subsubsection.}
\setcounter{secnumdepth}{3}
\setcounter{tocdepth}{4}


\begin{document}

% Capa
\thispagestyle{empty}
\begin{center}
    \vspace*{0.2cm}
    \textbf{\Large{Sistema de Informação para Grupos de Pesquisa}}\\
	
    \vspace*{1.2cm}
    \Large{Arquiteto: Jorge J. G. Leandro}\\
    \Large{Iteração 4 - Grupo 3}
    
    \vskip 2cm
	\textsc{
	MAC0332 - 2011\\[-0.25cm] 
          	Engenharia de Software\\[-0.25cm] 	
	IME - USP\\[-0.25cm]
	}
    
    \vskip 1.5cm
    \textbf{Descrições de Casos de Uso}\\
    Prof: Marco Gerosa\\

	
    \vskip 0.5cm
 %   \normalsize{São Paulo, Julho de 2008}
   {\normalsize São Paulo, \today}
\end{center}


\chapter[Caso de Uso]{Caso de uso: \bf Visitante lista Disciplinas oferecidas por um Grupo}
\label{cap:casodeuso}	

\section{Breve descrição}

O caso de uso \textbf{Visitante lista Disciplinas oferecidas por um Grupo} permite que um visitante solicite uma listagem das disciplinas oferecidas por um grupo de pesquisa.

\section{Breve Descrição dos Atores}

\subsection{Visitante}

O \textbf{Visitante} é o usuário de um computador com acesso à \emph{Internet}.

\section{Pré-condições}
O visitante tem acesso a um computador com software navegador e conexão à \emph{Internet}.
O navegador exibe a página de algum Grupo cadastrado no SIGP

\subsection{Fluxo Básico de Eventos}

\begin{enumerate}
\item O caso de uso começa quando o Visitante acessa a página de um GRUPO cadastrado no SIGP
\item O Visitante clica no botão listar disciplinas
\item O SIGP redireciona o visitante para outra página, contendo uma lista com as disciplinas oferecidas por aquele Grupo.
\item O caso de uso termina.
\end{enumerate}

\section{Fluxos Alternativos}

\subsection{Impressão da Lista}
Se no passo $3$ do fluxo básico, o Visitante desejar, pode solicitar a impressão da página, então

\begin{enumerate}
\item O Visitante aciona o botão de impressão
\item O SIGP apresenta uma página amigável de impressão contendo a lista desejada
\item O Visitante aciona o botão fechar página de impressão
\item O caso de uso retoma o passo 4
\end{enumerate}

\subsection{Nenhuma Disciplina é oferecida pelo Grupo}
Se no passo $3$ do fluxo básico, não houver disciplinas oferecidas pelo Grupo em questão, o SIGP exibe uma página com uma lista vazia e uma mensagem informando que este Grupo não oferece nenhuma disciplina no momento.

\begin{enumerate}
\item O Visitante aciona o botão de impressão
\item O SIGP apresenta uma página amigável de impressão contendo a lista desejada
\item O Visitante aciona o botão fechar página de impressão
\item O caso de uso retoma o passo 4
\end{enumerate}

%\section{Subfluxos}
%
%\subsection{Subfluxo 1}
%
%\begin{enumerate}
%\item subfluxo 1, passo 1
%\item $\ldots$
%\item subfluxo 1, passo n
%\end{enumerate}
%
%\section{Cenários Chaves}
%
%\subsection{Cenário 1}
%\begin{enumerate}
%\item cenário 1, passo 1
%\item $\ldots$
%\item cenário 1, passo n
%\end{enumerate}

\section{Pós-condições}

\subsection{Listagem bem sucedida}

O usuário visualiza a listagem e consegue imprimi-la, caso deseje.

%\section{Requisitos Especiais}
%
%Requisito especial 1

\chapter[Caso de Uso]{Caso de uso: \bf Visitante lista Grupos}
\label{cap:casodeuso}	

\section{Breve descrição}

O caso de uso \textbf{Visitante lista Grupos} permite que um visitante solicite uma listagem dos grupos de pesquisa cadastrados no SIGP.

\section{Breve Descrição dos Atores}

\subsection{Visitante}

O \textbf{Visitante} é o usuário de um computador com acesso à \emph{Internet}.

\section{Pré-condições}
O visitante tem acesso a um computador com software navegador e conexão à \emph{Internet}.
O navegador exibe a página inicial do SIGP

\subsection{Fluxo Básico de Eventos}

\begin{enumerate}
\item O caso de uso começa quando o Visitante acessa a inicial do SIGP
\item O Visitante clica no botão listar grupos
\item O SIGP redireciona o visitante para outra página, contendo uma lista com os grupos cadastrados no SIGP.
\item O caso de uso termina.
\end{enumerate}

\section{Fluxos Alternativos}

\subsection{Impressão da Lista}
Se no passo $3$ do fluxo básico, o Visitante desejar, pode solicitar a impressão da página, então

\begin{enumerate}
\item O Visitante aciona o botão de impressão
\item O SIGP apresenta uma página amigável de impressão contendo a lista desejada
\item O Visitante aciona o botão fechar página de impressão
\item O caso de uso retoma o passo 4
\end{enumerate}

\subsection{Nenhum Grupo está cadastrado no SIGP}
Se no passo $3$ do fluxo básico, não houver grupos cadastrados no sistema, o SIGP exibe uma página com uma lista vazia e uma mensagem informando não há grupos cadastrados até o momento.

\begin{enumerate}
\item O Visitante aciona o botão de impressão
\item O SIGP apresenta uma página amigável de impressão contendo a lista desejada
\item O Visitante aciona o botão fechar página de impressão
\item O caso de uso retoma o passo 4
\end{enumerate}

%\section{Subfluxos}
%
%\subsection{Subfluxo 1}
%
%\begin{enumerate}
%\item subfluxo 1, passo 1
%\item $\ldots$
%\item subfluxo 1, passo n
%\end{enumerate}
%
%\section{Cenários Chaves}
%
%\subsection{Cenário 1}
%\begin{enumerate}
%\item cenário 1, passo 1
%\item $\ldots$
%\item cenário 1, passo n
%\end{enumerate}

\section{Pós-condições}

\subsection{Listagem bem sucedida}

O usuário visualiza a listagem e consegue imprimi-la, caso deseje.

%\section{Requisitos Especiais}
%
%Requisito especial 1

\chapter[Caso de Uso]{Caso de uso: \bf Visitante lista linhas de pesquisa}
\label{cap:casodeuso}	

\section{Breve descrição}

O caso de uso \textbf{Visitante lista linhas de pesquisa} permite que um visitante liste linhas de pesquisa cadastradas no SIGP em dois pontos: Na página de um grupo são listadas as linhas de pesquisa associadas a este e na página de uma linha de pesquisa são listadas as linhas de pesquisa hierarquicamente inferiores a esta.

\section{Breve Descrição dos Atores}

\subsection{Visitante}

O \textbf{Visitante} é o usuário de um computador com acesso à \emph{Internet}.

\section{Pré-condições}
O visitante tem acesso a um computador com software navegador e conexão à \emph{Internet}.
O navegador exibe a página inicial do SIGP

\subsection{Fluxo Básico de Eventos}

\subsubsection{Através da página de um grupo}
\begin{enumerate}
\item O caso de uso começa quando o Visitante acessa a página de um grupo no SIGP;
\item O Visitante clica no botão listar linhas de pesquisa;
\item O SIGP redireciona o visitante para outra página, contendo uma lista com as linhas de pesquisa associadas a este no SIGP;
\item O caso de uso termina.
\end{enumerate}

\subsubsection{Através da página de uma linha de pesquisa}
\begin{enumerate}
\item O caso de uso começa quando o Visitante acessa a página de uma linha de pesquisa no SIGP;
\item O Visitante clica no botão listar linhas de pesquisa;
\item O SIGP redireciona o visitante para outra página, contendo uma lista com as linhas de pesquisa hierarquicamente inferiores a esta no SIGP;
\item O caso de uso termina.
\end{enumerate}

\section{Fluxos Alternativos}

\subsection{Impressão da Lista}
Se no passo $3$ do fluxo básico através da página de grupo, o Visitante desejar, pode solicitar a impressão da página, então

\begin{enumerate}
\item O Visitante aciona o botão de impressão
\item O SIGP apresenta uma página amigável de impressão contendo a lista desejada
\item O Visitante aciona o botão fechar página de impressão
\item O caso de uso retoma o passo 4
\end{enumerate}

\subsection{Nenhuma linha de pesquisa está associada ao grupo no SIGP}
Se no passo $3$ do fluxo básico, não houver linhas de pesquisa associadas ao grupo no sistema, o SIGP exibe uma página com uma lista vazia e uma mensagem informando.

\subsection{Nenhuma linha de pesquisa está abaixo à linha de pesquisa sendo vista no SIGP}
Se no passo $3$ do fluxo básico, não houver linhas de pesquisa hierarquicamente inferiores no sistema, o SIGP exibe uma página com uma lista vazia e uma mensagem informando.

%\section{Subfluxos}
%
%\subsection{Subfluxo 1}
%
%\begin{enumerate}
%\item subfluxo 1, passo 1
%\item $\ldots$
%\item subfluxo 1, passo n
%\end{enumerate}
%
%\section{Cenários Chaves}
%
%\subsection{Cenário 1}
%\begin{enumerate}
%\item cenário 1, passo 1
%\item $\ldots$
%\item cenário 1, passo n
%\end{enumerate}

\section{Pós-condições}

\subsection{Listagem bem sucedida}

O usuário visualiza a listagem e consegue imprimi-la, caso deseje.

%\section{Requisitos Especiais}
%
%Requisito especial 1

\chapter[Caso de Uso]{Caso de uso: \bf Visitante lista projetos}
\label{cap:casodeuso}	

\section{Breve descrição}

O caso de uso \textbf{Visitante lista projetos} permite que um visitante liste projetos cadastrados no SIGP em dois pontos: Na página de um grupo são listadas os projetos conduzidos por este e na página de uma linha de pesquisa são listadas os projetos conduzidos por esta.

\section{Breve Descrição dos Atores}

\subsection{Visitante}

O \textbf{Visitante} é o usuário de um computador com acesso à \emph{Internet}.

\section{Pré-condições}
O visitante tem acesso a um computador com software navegador e conexão à \emph{Internet}.
O navegador exibe a página inicial do SIGP

\subsection{Fluxo Básico de Eventos}

\subsubsection{Através da página de um grupo}
\begin{enumerate}
\item O caso de uso começa quando o Visitante acessa a página de um grupo no SIGP;
\item O Visitante clica no botão listar projetos;
\item O SIGP redireciona o visitante para outra página, contendo uma lista com os projetos conduzidos por este no SIGP;
\item O caso de uso termina.
\end{enumerate}

\subsubsection{Através da página de uma linha de pesquisa}
\begin{enumerate}
\item O caso de uso começa quando o Visitante acessa a página de uma linha de pesquisa no SIGP;
\item O Visitante clica no botão listar projetos;
\item O SIGP redireciona o visitante para outra página, contendo uma lista com os projetos conduzidos por esta no SIGP;
\item O caso de uso termina.
\end{enumerate}

\section{Fluxos Alternativos}

\subsection{Impressão da Lista}
Se no passo $3$ do fluxo básico através da página de grupo, o Visitante desejar, pode solicitar a impressão da página, então

\begin{enumerate}
\item O Visitante aciona o botão de impressão
\item O SIGP apresenta uma página amigável de impressão contendo a lista desejada
\item O Visitante aciona o botão fechar página de impressão
\item O caso de uso retoma o passo 4
\end{enumerate}

\subsection{Nenhuma linha de pesquisa está associada ao grupo no SIGP}
Se no passo $3$ do fluxo básico, não houver projetos conduzidos pelo grupo no sistema, o SIGP exibe uma página com uma lista vazia e uma mensagem informando.

\subsection{Nenhuma linha de pesquisa está abaixo à linha de pesquisa sendo vista no SIGP}
Se no passo $3$ do fluxo básico, não houver projetos conduzidos pela linha de pesquisa no sistema, o SIGP exibe uma página com uma lista vazia e uma mensagem informando.

%\section{Subfluxos}
%
%\subsection{Subfluxo 1}
%
%\begin{enumerate}
%\item subfluxo 1, passo 1
%\item $\ldots$
%\item subfluxo 1, passo n
%\end{enumerate}
%
%\section{Cenários Chaves}
%
%\subsection{Cenário 1}
%\begin{enumerate}
%\item cenário 1, passo 1
%\item $\ldots$
%\item cenário 1, passo n
%\end{enumerate}

\section{Pós-condições}

\subsection{Listagem bem sucedida}

O usuário visualiza a listagem e consegue imprimi-la, caso deseje.

%\section{Requisitos Especiais}
%
%Requisito especial 1

\chapter[Caso de Uso]{Caso de uso: \bf Visitante lista publicações}
\label{cap:casodeuso}	

\section{Breve descrição}

O caso de uso \textbf{Visitante lista publicações} permite que um visitante liste publicações cadastradas no SIGP em três pontos: Na página de um grupo são listados todos as publicações deste; na página de uma linha de pesquisa são listadas todos as publicações dos contribuintes desta; e na página de um projeto são listados todas as publicações dos contribuintes deste.

\section{Breve Descrição dos Atores}

\subsection{Visitante}

O \textbf{Visitante} é o usuário de um computador com acesso à \emph{Internet}.

\section{Pré-condições}
O visitante tem acesso a um computador com software navegador e conexão à \emph{Internet}.
O navegador exibe a página inicial do SIGP

\subsection{Fluxo Básico de Eventos}

\subsubsection{Através da página de um grupo}
\begin{enumerate}
\item O caso de uso começa quando o Visitante acessa a página de um grupo no SIGP;
\item O Visitante clica no botão listar publicações;
\item O SIGP redireciona o visitante para outra página, contendo uma lista com as publicações dos contribuintes participando deste no SIGP;
\item O caso de uso termina.
\end{enumerate}

\subsubsection{Através da página de uma linha de pesquisa}
\begin{enumerate}
\item O caso de uso começa quando o Visitante acessa a página de uma linha de pesquisa no SIGP;
\item O Visitante clica no botão listar contribuintes;
\item O SIGP redireciona o visitante para outra página, contendo uma lista com as publicações dos contribuintes participando desta no SIGP;
\item O caso de uso termina.
\end{enumerate}

\subsubsection{Através da página de um projeto}
\begin{enumerate}
\item O caso de uso começa quando o Visitante acessa a página de um projeto no SIGP;
\item O Visitante clica no botão listar contribuintes;
\item O SIGP redireciona o visitante para outra página, contendo uma lista com as publicações dos contribuintes participando deste no SIGP;
\item O caso de uso termina.
\end{enumerate}

\section{Fluxos Alternativos}

\subsection{Impressão da Lista}
Se no passo $3$ do fluxo básico através da página de grupo, o Visitante desejar, pode solicitar a impressão da página, então

\begin{enumerate}
\item O Visitante aciona o botão de impressão
\item O SIGP apresenta uma página amigável de impressão contendo a lista desejada
\item O Visitante aciona o botão fechar página de impressão
\item O caso de uso retoma o passo 4
\end{enumerate}

\subsection{Nenhum contribuinte que publicou está no grupo no SIGP}
Se no passo $3$ do fluxo básico, não houver contribuintes que publicaram no grupo no sistema, o SIGP exibe uma página com uma lista vazia e uma mensagem informando.

\subsection{Nenhum contribuinte que publicou está na linha de pesquisa sendo vista no SIGP}
Se no passo $3$ do fluxo básico, não houver contribuintes que publicaram na linha de pesquisa no sistema, o SIGP exibe uma página com uma lista vazia e uma mensagem informando.

\subsection{Nenhum contribuinte que publicou faz parte do projeto sendo visto no SIGP}
Se no passo $3$ do fluxo básico, não houver contribuintes que publicaram fazendo parte do projeto sendo visto no sistema, o SIGP exibe uma página com uma lista vazia e uma mensagem informando.

%\section{Subfluxos}
%
%\subsection{Subfluxo 1}
%
%\begin{enumerate}
%\item subfluxo 1, passo 1
%\item $\ldots$
%\item subfluxo 1, passo n
%\end{enumerate}
%
%\section{Cenários Chaves}
%
%\subsection{Cenário 1}
%\begin{enumerate}
%\item cenário 1, passo 1
%\item $\ldots$
%\item cenário 1, passo n
%\end{enumerate}

\section{Pós-condições}

\subsection{Listagem bem sucedida}

O usuário visualiza a listagem e consegue imprimi-la, caso deseje.

%\section{Requisitos Especiais}
%
%Requisito especial 1

\chapter[Caso de Uso]{Caso de uso: \bf Visitante lista contribuintes}
\label{cap:casodeuso}	

\section{Breve descrição}

O caso de uso \textbf{Visitante lista contribuinte} permite que um visitante liste projetos cadastrados no SIGP em três pontos: Na página de um grupo são listados todos os contribuintes deste; na página de uma linha de pesquisa são listadas todos os contribuintes desta; e na página de um projeto são listados todos os contribuintes deste.

\section{Breve Descrição dos Atores}

\subsection{Visitante}

O \textbf{Visitante} é o usuário de um computador com acesso à \emph{Internet}.

\section{Pré-condições}
O visitante tem acesso a um computador com software navegador e conexão à \emph{Internet}.
O navegador exibe a página inicial do SIGP

\subsection{Fluxo Básico de Eventos}

\subsubsection{Através da página de um grupo}
\begin{enumerate}
\item O caso de uso começa quando o Visitante acessa a página de um grupo no SIGP;
\item O Visitante clica no botão listar contribuintes;
\item O SIGP redireciona o visitante para outra página, contendo uma lista com os contribuintes participando deste no SIGP;
\item O caso de uso termina.
\end{enumerate}

\subsubsection{Através da página de uma linha de pesquisa}
\begin{enumerate}
\item O caso de uso começa quando o Visitante acessa a página de uma linha de pesquisa no SIGP;
\item O Visitante clica no botão listar contribuintes;
\item O SIGP redireciona o visitante para outra página, contendo uma lista com os contribuintes participando desta no SIGP;
\item O caso de uso termina.
\end{enumerate}

\subsubsection{Através da página de um projeto}
\begin{enumerate}
\item O caso de uso começa quando o Visitante acessa a página de um projeto no SIGP;
\item O Visitante clica no botão listar contribuintes;
\item O SIGP redireciona o visitante para outra página, contendo uma lista com os contribuintes participando deste no SIGP;
\item O caso de uso termina.
\end{enumerate}

\section{Fluxos Alternativos}

\subsection{Impressão da Lista}
Se no passo $3$ do fluxo básico através da página de grupo, o Visitante desejar, pode solicitar a impressão da página, então

\begin{enumerate}
\item O Visitante aciona o botão de impressão
\item O SIGP apresenta uma página amigável de impressão contendo a lista desejada
\item O Visitante aciona o botão fechar página de impressão
\item O caso de uso retoma o passo 4
\end{enumerate}

\subsection{Nenhum contribuinte está no grupo no SIGP}
Se no passo $3$ do fluxo básico, não houver contribuintes no grupo no sistema, o SIGP exibe uma página com uma lista vazia e uma mensagem informando.

\subsection{Nenhum contribuinte está na linha de pesquisa sendo vista no SIGP}
Se no passo $3$ do fluxo básico, não houver contribuintes na linha de pesquisa no sistema, o SIGP exibe uma página com uma lista vazia e uma mensagem informando.

\subsection{Nenhum contribuinte faz parte do projeto sendo visto no SIGP}
Se no passo $3$ do fluxo básico, não houver contribuintes fazendo parte do projeto sendo visto no sistema, o SIGP exibe uma página com uma lista vazia e uma mensagem informando.

%\section{Subfluxos}
%
%\subsection{Subfluxo 1}
%
%\begin{enumerate}
%\item subfluxo 1, passo 1
%\item $\ldots$
%\item subfluxo 1, passo n
%\end{enumerate}
%
%\section{Cenários Chaves}
%
%\subsection{Cenário 1}
%\begin{enumerate}
%\item cenário 1, passo 1
%\item $\ldots$
%\item cenário 1, passo n
%\end{enumerate}

\section{Pós-condições}

\subsection{Listagem bem sucedida}

O usuário visualiza a listagem e consegue imprimi-la, caso deseje.

%\section{Requisitos Especiais}
%
%Requisito especial 1


\chapter[Caso de Uso]{Caso de uso: \bf Usuário faz login no SIGP}
\label{cap:casodeuso}	

\section{Breve descrição}

O caso de uso \textbf{Usuário faz login no SIGP} permite que um visitante se autentique como um usuário no sistema.

\section{Breve Descrição dos Atores}

\subsection{Visitante}

O \textbf{Visitante} é o usuário de um computador com acesso à \emph{Internet}.

\subsection{Usuário}

O \textbf{Usuário} é um \emph{Visitante} já autenticado no sistema.

\section{Pré-condições}
O visitante tem acesso a um computador com software navegador e conexão à \emph{Internet}.
O navegador exibe a página de login do SIGP.

\subsection{Fluxo Básico de Eventos}

\begin{enumerate}
\item O caso de uso começa quando o Visitante acessa a página de login do SIGP
\item O Visitante preenche o formulário com seu login e senha e então clica em login.
\item O SIGP redireciona o visitante para a página principal, onde parte da página exibe o nome do Usuário.
\item O caso de uso termina.
\end{enumerate}

\section{Fluxos Alternativos}

\subsection{Usuário e/ou Senha inválidos}
Se no passo $2$ do fluxo básico, o login e senha digitados corresponder a nenhum usuário do sistema, o SIGP exibe uma mensagem de erro ao invéz de redirecionar o usuário.

%\section{Subfluxos}
%
%\subsection{Subfluxo 1}
%
%\begin{enumerate}
%\item subfluxo 1, passo 1
%\item $\ldots$
%\item subfluxo 1, passo n
%\end{enumerate}
%
%\section{Cenários Chaves}
%
%\subsection{Cenário 1}
%\begin{enumerate}
%\item cenário 1, passo 1
%\item $\ldots$
%\item cenário 1, passo n
%\end{enumerate}

\section{Pós-condições}

\subsection{Login bem sucedido}

O usuário está autenticado e pode usar recursos que exigem permissões maiores do que a de um visitante.

%\section{Requisitos Especiais}
%
%Requisito especial 1


\end{document}

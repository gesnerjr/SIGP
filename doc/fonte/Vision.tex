\documentclass[11pt, a4paper]{article}

\usepackage[brazil]{babel}
\usepackage[utf8]{inputenc}
\usepackage[T1]{fontenc}
\usepackage[pdftex]{hyperref}
\usepackage{graphicx}
\usepackage{amsmath}
\usepackage{indentfirst}
\usepackage{fancyhdr}


% Formatação
\topmargin -1.5cm
\oddsidemargin -0.04cm
\evensidemargin -0.04cm
\textwidth 16.59cm
\textheight 21.94cm 
%\pagestyle{empty}                     % Sem numero de paginas
\pagestyle{fancy}                         %cabecalhos e rodares
\fancyhead[RO,RE]{\today}
\fancyhead[LO,LE]{MAC0332 - SI para grupos de pesquisa\\
	Vision}
\fancyfoot[LO,LE]{Confidential}
\fancyfoot[RO,RE] {\thepage}
\fancyfoot[CO,CE]{Grupo 3, 2011}


\parskip 7.2pt                        % Espaço entre paragrafos 7.2
%\renewcommand{\baselinestretch}{1.5} % Espaçamento entre linhas = 1.5
%\parindent 0pt

% Tirar hifenização
\hyphenpenalty = 5000
\tolerance = 1000
\sloppy

\title{MAC 0332\\
	Engenharia de Software\\
	SI para grupos de pesquisa\\
	Vision}
\date{\today}

\begin{document}

	\maketitle
	\newpage
    \section{Introdução}
        Nosso projeto visa sistematizar a estrutura de grupos de pesquisa em um contexto acadêmico
        de maneira que seja fácil identificar e relacionar esses grupos com seus membros, suas
        linhas de pesquisas, seus projetos e suas publicações. Além disso, visa também apresentar
        algumas funcionalidades básicas de sistemas WEB como: cadastros de usuários, possibilidade
        de comentar sobre publicações e fazer \textit{upload} de PDFs.
    
    \section{Posicionamento}
        Devido ao forte caráter hierárquico do sistema de grupos de pesquisa acadêmicos, a solução
        apresentada pelo nosso projeto foca na listagens de elementos a partir de algum outro
        elemento que lhe seja superior nessa hierarquia. Por exemplo, a partir de um grupo de
        pesquisa, nosso sistema será capaz de listar todos os subgrupos de pesquisa desse grupo e
        todas as linhas de pesquisa sobre as quais ele opera.
        
        No caso específico das linhas de pesquisa, que apresentam uma hierarquia própria, além de
        ser possível listá-las, o sistema deverá ser capaz de descrever, de alguma maneira, essas
        relações hierárquicas das linhas de pesquisa em que cada grupo de pesquisa trabalha.
        
        \subsection{Declaração de problema}
            %[Provide a statement summarizing the problem being solved by 
            %this project. The following format may be used:]
            \begin{tabular}{| l | l |}
                \hline                       
                O problema de & Disponibilizar informações sobre grupos de pesquisa \\ \hline
                Afeta & Quem publica algo num certo grupo de pesquisa \\ \hline
                O impacto do qual é & Não ter meios para categorizar e disponibilizar\\ 
                                                 &  informações sobre pesquisas \\ \hline
                Uma solução bem sucedida seria & Disponibilizar recursos para divulgar \\
                                                                    & organizadamente informações sobre pesquisas \\
                \hline  
            \end{tabular}
        
        \subsection{Declaração de Posição do Produto}
            %[Provide an overall statement summarizing, at the highest level, 
            %the unique position the product intends to fill in the marketplace. 
            %The following format may be used:]
            \begin{tabular}{| l | l |}
                \hline                       
                Para & Acadêmicos \\ \hline
                Que  & precisam de recursos para divulgação de informações sobre pesquisa \\ \hline
                O SIGP é um & Website \\ \hline
                Que & Provê todos os recursos para divulgar organizadamente pesquisas. \\
                \hline
            \end{tabular}
    \section{Descrição dos Envolvidos}
        %[Detail the working environment of the target user. Here are some suggestions:
        %Number of people involved in completing the task? Is this changing?
        %How long is a task cycle? Amount of time spent in each activity? Is this changing?
        %Any unique environmental constraints: mobile, outdoors, in-flight, and so on?
        %Which system platforms are in use today? Future platforms?
        %What other applications are in use? Does your application need to integrate with them?
        %This is where extracts from the Business Model could be included to outline the task and roles involved, and so on.]

\subsection{Resumo sobre envolvidos}

           \begin{tabular}{| l | l | l|}
                \hline
                Nome &                      Descrição &              Responsabilidades \\ \hline
                Rafael Reggiani Manzo &     Desenvolvedor &          Model e testes\\  \hline
                Evandro Giovanini &         Desenvolvedor &          Interface do Usuário\\ \hline
                Geraldo Castro Zampoli  &   Desenvolvedor &          Model e testes\\ \hline
                Henrique G Passos Lima &    Arquiteto &              Soluções de modelagem,\\
                                        &                       &    Controllers, Views e testes\\ \hline
                Jorge J G Leandro  &        Desenvolvedor &          Testes e Projeto BD \\  \hline
                Samuel Placa &              Analista de Qualidade &  Qualidade do código e testes \\ \hline
                Wilson Kazuo Mizutani  &    Analista de Requisitos & Negociação com cliente  e\\
                                   &                            &    Projeto BD\\ \hline
                Fernando Omar Aluani   &    Gerente &                Gerenciamento dos membros, \\
                                   &                            &    Controllers, Views e testes\\ \hline
                \hline
          \end{tabular}
         
  \subsection{Ambiente de Usuário}

    O ambiente do usuário poderá ser qualquer computador pessoal capaz de rodar um navegador WEB
    moderno (IE8 ou mais recente, Mozilla Firefox e afins, Chromium e derivados, Safari, Opera,
    etc). A interface para o usuário do sistema será totalmente disponibilizado através de um
    \textit{Web Site}.
    
    \section{Visão Geral do Produto}
        \subsection{Necessidades e recursos}
            \begin{tabular}{| l | c | p{5cm} | c |}
                \hline
                Necessidade & Prioridade & Recurso & Lançamento planejado \\ \hline
                Parte Pública & Alta & Listar disciplina, projetos, publicações, linhas de pesquisa, membros do grupo & ???\\ \hline
                Adminstrador & Alta & cadastrar um grupo, que 
                pode ser subgrupo de um grupo maior & ???\\ \hline
                Membro & Alta & Cadastrar publicação e fazer 
                \emph{upload} de artigo em formato PDF & ???\\ \hline
                Projeto & Alta & Descrição e membros com suas 
                publicações & ???\\ \hline
                Linha de Pesquisa & Alta & Descrição, publicações, 
                membros e projetos associados & ???\\ \hline
                Projeto & Alta & Descrição e membros com suas 
                publicações & ???\\ \hline
                \hline
            \end{tabular}
                
            


    
\end{document}

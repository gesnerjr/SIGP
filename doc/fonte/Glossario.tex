
\documentclass[11pt, a4paper]{article}

\usepackage[brazil]{babel}
\usepackage[utf8]{inputenc}
\usepackage[T1]{fontenc}
\usepackage[pdftex]{hyperref}
\usepackage{graphicx}
\usepackage{amsmath}
\usepackage{indentfirst}
\usepackage{fancyhdr}

% Formatação
\topmargin -1.5cm
\oddsidemargin -0.04cm
\evensidemargin -0.04cm
\textwidth 16.59cm
\textheight 21.94cm 
%\pagestyle{empty}                     % Sem numero de paginas
\pagestyle{fancy}                         %cabecalhos e rodares
\fancyhead[RO,RE]{\today}
\fancyhead[LO,LE]{MAC0332 - SI para grupos de pesquisa\\
	Glossário}
\fancyfoot[LO,LE]{Confidential}
\fancyfoot[RO,RE] {\thepage}
\fancyfoot[CO,CE]{Grupo 3, 2011}


\parskip 7.2pt                        % Espaço entre paragrafos 7.2
%\renewcommand{\baselinestretch}{1.5} % Espaçamento entre linhas = 1.5
%\parindent 0pt

% Tirar hifenização
\hyphenpenalty = 5000
\tolerance = 1000
\sloppy

\title{MAC 0332\\
	Engenharia de Software\\
	SI para grupos de pesquisa\\
	Glossário}
\date{\today}

\begin{document}

	\maketitle
	\newpage

	\noindent\textbf{\huge{A}}\\
	\line(1,0){450}\\
	\textbf{Administrador}: Um tipo especial de usuário, o único 
	que pode inscrever um novo grupo, uma nova linha de pesquisa
	e um novo contribuinte.\\
	\textbf{Annotation}: Definições especiais para variaveis e 
	classes em Java.\\
	\textbf{Arcabouço}: Framework.\\
	
	\noindent\textbf{\huge{B}}\\
	\line(1,0){450}\\
	\textbf{Banco de Dados}: Repositório de dados persistentes gravado em disco.\\

	\noindent\textbf{\huge{C}}\\
	\line(1,0){450}\\
	\textbf{Contribuinte}: Aquele que contribui seja como filiado a um grupo de pesquisa ou como
	autor de alguma publicação.\\
	\textbf{CSS}: Cascading Style Sheets - linguagem de estilo utilizada para descrever a semântica de apresentação de documentos escritos em uma linguagem de marcação, como HTML ou XML.\\

	\noindent\textbf{\huge{D}}\\
	\line(1,0){450}\\
	\textbf{DER}: Diagrama de Entidade-Relacionamenro.\\
	\textbf{Departamento}: a menor fração da estrutura universitária para os efeitos de organização didático-científica e administrativa. Pode conter muitos grupos de pesquisa.\\
	\textbf{Disciplina}: estudo de um ramo do saber humano. Um departamento pode ministrar várias disciplinas dentro da área de sua especialidade.\\
	\textbf{DRY}: Don't Repeat Yourseft.\\
	
	\noindent\textbf{\huge{E}}\\
	\line(1,0){450}\\
	\textbf{Eclipse}: Ambiente Integrado para Desenvolvimento (Integrated Development Environment - IDE).\\
	\textbf{Entidade}: Objeto básico do modelo Entidade-Relacionamento que representa um objeto do mundo real, com existência independente e é representado por uma relação (tabela) no Banco de Dados.\\
	
	\noindent\textbf{\huge{F}}\\
	\line(1,0){450}\\
	\textbf{Filiação}: Qualidade do contribuinte que está filiado a um grupo de pesquisa.\\
	\textbf{Framework}: Arcabouço.\\

	\noindent\textbf{\huge{G}}\\
	\line(1,0){450}\\
	\textbf{Grupo de pesquisa}: um conjunto de indivíduos organizados em torno de um ou mais objetos de estudo. A liderança do grupo ocorre por um docente da USP, com titulação mínima de doutor. Neste sistema, somente o administrador pode criar um novo grupo.\\
	
	\noindent\textbf{\huge{H}}\\
	\line(1,0){450}\\
	\textbf{Hibernate}: um \emph{framework} Java para persistência. Desempenha o mapeamento objeto-relacional e consultas em bancos de dados, por meio das linguagens HQL e SQL.\\
	\textbf{HTML}: HyperText Markup Language - é uma linguagem de marcação 
	utilizada para produzir páginas na Web.\\
	
	\noindent\textbf{\huge{I}}\\
	\line(1,0){450}\\
	\textbf{IDE}: \emph{Integrated Development Environment}, um ambiente integrado 
	para desenvolvimento de software.\\
	
	\noindent\textbf{\huge{J}}\\
	\line(1,0){450}\\
	\textbf{Java}: Linguagem de programação orientada a objetos, que propicia o desenvolvimento de apllicações multiplataformas, por ser compilável em \emph{bytecodes}, os quais são interpretáveis por máquinas virtuais.\\
	\textbf{JSP}: JavaServer Pages - é uma tecnologia utilizada no 
	desenvolvimento de aplicações em Java para Web, com páginas geradas e servidas dinamicamente.\\
	\textbf{JUnit}: \emph{framework} de código aberto que provê suporte para criação de testes automatizados, em particular, Testes de Unidade em Java.\\
	
	\noindent\textbf{\huge{K}}\\
	\line(1,0){450}\\
		
	\noindent\textbf{\huge{L}}\\
	\line(1,0){450}\\
	\textbf{Linhas de Pesquisa}: São campos temáticos que delimitam os objetos de estudos e pesquisas em um ambiente acadêmico.
	São conduzidas por um grupo de pesquisa.\\
		
	\noindent\textbf{\huge{M}}\\
	\line(1,0){450}\\
	\textbf{Membros}: Membros de um grupo de pesquisa. Termo depreciado, ver Contribuinte,
	Filiação e Usuário.\\
	\textbf{MER}: Modelo Entidade Relacionamento.\\
	\textbf{Mockito}: \emph{framework} de código aberto que provê suporte para criação de testes automatizados, em particular, Testes de Unidade em Java, permitindo a criação de objetos \emph{mock}, que simulam objetos reais.\\
	\textbf{MVC}: Model-View-Controller - é um padrão de arquitetura de software
	que visa a separar a lógica de negócio da lógica de apresentação, permitindo
	o desenvolvimento, teste e manutenção isolado de ambos.\\
	\textbf{MySQL}: Banco de Dados relacional de código aberto.\\
			
	\noindent\textbf{\huge{N}}\\
	\line(1,0){450}\\
	
	\noindent\textbf{\huge{O}}\\
	\line(1,0){450}\\
	\textbf{OpenUp}: Processo Unificado Aberto - \emph{framework} de processos que fornece as
	melhores práticas colhidas entre opinões de líderes em desenvolvimento de software e aplica
	abordagens incremental e iterativa dentro de um ciclo de vida estruturado.\\
	
	\noindent\textbf{\huge{P}}\\
	\line(1,0){450}\\
	\textbf{Pencil}: Aplicativo para desenho de telas.\\
	\textbf{Projetos}: um texto que define e detalha as minúcias do planejamento para a execução de
	um trabalho científico de pesquisa. Estão associados a uma ou mais linhas de pesquisa.\\
	\textbf{Publicação}: trabalho escrito na forma de artigo para descrição e registro de uma
	produção científica, artística ou tecnológica. Podem estar associadas a diversos projetos.\\
	\textbf{Público}: a parte do sistema dispobilizada para visualização por todo visitante
	da comunidade de internautas.\\
	
	\noindent\textbf{\huge{Q}}\\
	\line(1,0){450}\\
	\textbf{Query}: Consulta ao Banco de Dados.\\
	
	\noindent\textbf{\huge{R}}\\
	\line(1,0){450}\\
	
	\noindent\textbf{\huge{S}}\\
	\line(1,0){450}\\
	\textbf{Subgrupo de pesquisa}: equipe de membros que compõem um subconjunto dos membros de um grupo de 
	pesquisa; somente o admininstrador pode criar um subgrupo de um grupo.\\
		
	\noindent\textbf{\huge{T}}\\
	\line(1,0){450}\\
		
	\noindent\textbf{\huge{U}}\\
	\line(1,0){450}\\
	\textbf{Usuário}: São os usuários cadastrados no sistema. Ou correspondem a um contribuinte
	ou são um administrador - um tipo especial de usuário. Quando correspondem a um contribuinte,
	podem propor novos projetos nas linhas de pesquisa em que estudam e podem registrar publicações
	no sistema.\\
		
	\noindent\textbf{\huge{V}}\\
	\line(1,0){450}\\
	\textbf{VRaptor}: \emph{framework} MVC para desenvolvimento Web com Java, focado no
	desenvolvimento ágil e no conceito DRY.\\
	
	\noindent\textbf{\huge{W}}\\
	\line(1,0){450}\\
		
	\noindent\textbf{\huge{X}}\\
	\line(1,0){450}\\
	\textbf{XHTML}: eXtensible Hypertext Markup Language - é uma reformulação 
	da linguagem de marcação HTML, baseada em XML, visando a apresentação destas páginas em diversos tipos de mídia, priviliegiando a acessibilidade .\\
	\textbf{XML}: eXtensible Markup Language - um conjunto de regras para codificação de documentos na forma legível por máquinas, definida na especificação XML 1.0 produzida pela W3C.\\
	
	\noindent\textbf{\huge{Y}}\\
	\line(1,0){450}\\
		
	\noindent\textbf{\huge{Z}}\\
	\line(1,0){450}\\
	
\end{document}


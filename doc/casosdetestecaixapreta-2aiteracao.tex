\documentclass[11pt, a4paper]{book}

\usepackage[brazil]{babel}
\usepackage[utf8]{inputenc}
\usepackage[T1]{fontenc}
%\usepackage[pdftex]{hyperref}
\usepackage[pdftex]{graphicx}  

\usepackage{amsmath}
\usepackage{indentfirst}
\usepackage{fancyhdr}
\usepackage{setspace}         
\usepackage[pagebackref,colorlinks=true,urlcolor=cyan,citecolor=red,linkcolor=blue]{hyperref}

\usepackage{color}
\usepackage{pifont}
\usepackage{amsfonts}
\usepackage{amssymb}
\usepackage{amsmath}
\usepackage{setspace}                  
\usepackage[small,compact]{titlesec} 		

\usepackage[square,sort,nonamebreak]{natbib}
\usepackage[Lenny]{fncychap}					
\usepackage{url}
\usepackage{latexsym}
\usepackage{multicol}
\usepackage{multirow}

% Formatação
\topmargin -1.5cm
\oddsidemargin -0.04cm
\evensidemargin -0.04cm
\textwidth 16.59cm
\textheight 21.94cm 
%\pagestyle{empty}                     % Sem numero de paginas
\pagestyle{fancy}                         %cabecalhos e rodares
\fancyhead[RO,RE]{\today}
\fancyhead[LO,LE]{MAC0332 - SI para grupos de pesquisa\\
	Casos de Teste Caixa Preta para Análise Funcional}
\fancyfoot[LO,LE]{Confidencial}
\fancyfoot[RO,RE] {\thepage}
\fancyfoot[CO,CE]{Grupo 3, 2011}


\parskip 7.2pt                        % Espaço entre paragrafos 7.2
%\renewcommand{\baselinestretch}{1.5} % Espaçamento entre linhas = 1.5
%\parindent 0pt

% Tirar hifenização
\hyphenpenalty = 5000
\tolerance = 1000
\sloppy


% numeração para subsubsection
%errado: \renewcommand\thesubsubsection{\@arabic\c@section.\@arabic\c@subsection.\@arabic\c@subsubsection.}
\setcounter{secnumdepth}{3}
\setcounter{tocdepth}{4}


\begin{document}

% Capa
\thispagestyle{empty}
\begin{center}
    \vspace*{0.2cm}
    \textbf{\Large{Sistema de Informação para Grupos de Pesquisa}}\\
	
    \vspace*{1.2cm}

    \Large{Desenvolvedor - Jorge J. G. Leandro}\\
    \Large{Grupo 3 - Iteração 2}
    
    \vskip 2cm
	\textsc{
	MAC0332 - 2011\\[-0.25cm] 
          	Engenharia de Software\\[-0.25cm] 	
	IME - USP\\[-0.25cm]
	}
    
    \vskip 1.5cm
    Casos de Teste Caixa Preta\\
    Prof: Marco Gerosa\\

	
    \vskip 0.5cm
 %   \normalsize{São Paulo, Julho de 2008}
   {\normalsize São Paulo, \today}
\end{center}


\chapter[Apresentação]{Apresentação}
\label{cap:apresentacao}	
	
	 	O presente documento descreve o cumprimento da tarefa \emph{Criar casos de teste}, mediante uma coleção de Casos de Teste Caixa Preta para Análise Funcional para o SIGP, conforme o documento de requisitos do sistema e de acordo com diretrizes de [Schach, 2007] e gabaritos do \emph{Processo Unificado Aberto - OpenUP}.

\chapter[Casos de Teste]{Casos de Teste Caixa Preta - Análise Funcional}
\label{cap:casosdeteste}

\section{Testar o cadastro de novos ítens no sistema.}

\begin{itemize}
	\item Caso de Teste - SIGP-CT-001
	\begin{itemize}
	\item Descrição: Cadastrar um \textbf{contribuinte}.
	\item Pré-condições: Ser administrador. Ter uma \emph{View} com formulário para cadastro de contribuinte. Banco de Dados criado.
	\item Pós-condições: O registro de mais um contribuinte no BD e apresentação na listagem subsequente.
	\item Dados necessários: Nome, Usuario, Publicações, Filiacao
	\end{itemize}

	\item  Caso de Teste - SIGP-CT-002 
	\begin{itemize}
	\item Descrição: Cadastrar um \textbf{usuário}.
	\item Pré-condições: Ser administrador. Ter uma \emph{View} com formulário para cadastro de usuario. Banco de Dados criado.
	\item Pós-condições: O registro de mais um usuário no BD e na listagem subsequente.
	\item Dados necessários: Login, Senha, Tipo, Avatar, Contribuinte
	\end{itemize}
	
	\item Caso de Teste - SIGP-CT-003
	\begin{itemize}
	\item Descrição: Cadastrar um \textbf{grupo}.
	\item Pré-condições: Ser administrador. Ter uma \emph{View} com formulário para cadastro de grupo. Ter o Banco de Dados criado.
	\item Pós-condições: O registro de mais um grupo no BD e na listagem subsequente.
	\item Dados necessários: Nome, Subgrupo, Lista de linhas de pesquisa, Lista de filiações, Lista de Disciplinas.
	\end{itemize}

\newpage

	\item Caso de Teste - SIGP-CT-004
	\begin{itemize}
	\item Descrição: Cadastrar um \textbf{subgrupo}.
	\item Pré-condições: Ser administrador. Ter uma \emph{View} com formulário para cadastro de subgrupo de um grupo.Ter o Banco de Dados criado.
	\item Pós-condições: O registro de mais um subgrupo no BD e na listagem subsequente.
	\item Dados necessários: Nome, Nome do grupo, Lista de linhas de pesquisa, Lista de filiações, Lista de disciplinas.
	\end{itemize}

	\item Caso de Teste - SIGP-CT-005
	\begin{itemize}
	\item Descrição: Cadastrar uma \textbf{publicação}.
	\item Pré-condições: Ser membro-usuário. Ter uma \emph{View} com formulário para cadastro de publicação.Ter o Banco de Dados criado.
	\item Pós-condições: O registro de mais uma publicação no BD e na listagem subsequente.
	\item Dados necessários: Titulo, Veiculo, Data, Projeto, Lista de Contribuintes
	\end{itemize}


	\item Caso de Teste - SIGP-CT-006
	\begin{itemize}
	\item Descrição: Cadastrar uma \textbf{linha de pesquisa}.
	\item Pré-condições: Ter uma \emph{View} com formulário para cadastrar linha de pesquisa. Banco de Dados criado.
	\item Pós-condições: O registro de mais uma linha de pesquisa no BD e na listagem subsequente.
	\item Dados necessários: Nome. Lista de projetos na linha de pesquisa. Lista de Grupos.
	\end{itemize}

	\item Caso de Teste - SIGP-CT-007
	\begin{itemize}
	\item Descrição: Cadastrar um \textbf{projeto}.
	\item Pré-condições: Banco de Dados criado.
	\item Pós-condições: O registro de mais um projeto no BD e na listagem subsequente.
	\item Dados necessários: Nome do projeto, Descrição do projeto, Nome da Agência de Fomento financiadora, Linha de pesquisa, Lista de publicações, llista de pessoas filiações.
	\end{itemize}

\newpage
\section{Testar o acréscimo de informações a ítens já cadastrados no sistema}

	\item Caso de Teste - SIGP-CT-008
	\begin{itemize}
	\item Descrição: Relacionar uma \textbf{publicação} aos \textbf{projetos} correlatos.
	\item Pré-condições: Ser usuário. Ter uma \emph{View} com campos para relacionar publicações a projetos.Ter o Banco de Dados criado.
	\item Pós-condições: O registro de mais uma publicação no BD e na listagem subsequente.
	\item Dados necessários: Título, veículo, Data, Lista de projetos, Lista de contribuintes
	\end{itemize}

	\item Caso de Teste - SIGP-CT-009
	\begin{itemize}
	\item Descrição: Fazer \emph{upload} de arquivo \emph{.pdf} de uma \textbf{publicação}.
	\item Pré-condições: Ser usuário. Ter uma \emph{View} com entrada para caminho do arquivo. Banco de Dados criado.
	\item Pós-condições: Exibir um ícone com um \emph{link} apontando para o arquivo recém enviado, ao lado da publicação em questão, na lista de publicações.
	\item Dados necessários: Nome e/ou ID da publicação. Arquivo \emph{.pdf} da publicação. 
	\end{itemize}

\newpage
\section{Testar a exibição de informações de ítens no sistema}

	\item Caso de Teste - SIGP-CT-010
	\begin{itemize}
	\item Descrição: Listar disciplinas oferecidas por um \textbf{grupo}.
	\item Pré-condições: Ser usuário. Ter uma \emph{View} para exibir uma listagem de disciplinas ministradas por um grupo.
	\item Pós-condições: Exibir a listagem das disciplinas na \emph{View}.
	\item Dados necessários: Nome  e/ou ID do grupo.
	\end{itemize}
    
	\item Caso de Teste - SIGP-CT-011
	\begin{itemize}
	\item Descrição: Listar projetos sob uma \textbf{linha de pesquisa} conduzidos por um \textbf{grupo}.
	\item Pré-condições: Ser usuário. Ter uma \emph{View} para exibir uma listagem de projetos coordenados por um grupo.
	\item Pós-condições: Exibir a listagem dos projetos na \emph{View}.
	\item Dados necessários: Nome  e/ou ID do grupo.
	\end{itemize}

	\item Caso de Teste - SIGP-CT-012
	\begin{itemize}
	\item Descrição: Listar \textbf{publicações} de um grupo, separadas por tipo.
	\item Pré-condições: Ser usuário. Ter uma \emph{View} para exibir uma listagem de publicações de um grupo.
	\item Pós-condições: Exibir a listagem de publicações separadas por tipo na \emph{View}.
	\item Dados necessários: Nome e/ou ID do grupo.
	\end{itemize}

	\item Caso de Teste - SIGP-CT-013
	\begin{itemize}
	\item Descrição: Listar filiações de um \textbf{grupo}.
	\item Pré-condições: Ser usuário. Ter uma \emph{View} para exibir uma listagem de filliações de um grupo.
	\item Pós-condições: Exibir a listagem de filiações de um grupo na \emph{View}.
	\item Dados necessários: Nome  e/ou ID do grupo.
	\end{itemize}

	\item Caso de Teste - SIGP-CT-014
	\begin{itemize}
	\item Descrição: Listar \textbf{linhas de pesquisa} de um grupo, organizadas hierarquicamente.
	\item Pré-condições: Ser membro-usuário. Ter uma \emph{View} para exibir uma listagem das linhas de pesquisas de um grupo. Banco de Dados criado.
	\item Pós-condições: Exibir uma listagem das linhas de pesquisas de um grupo, ordenadas hierarquicamente.
	\item Dados necessários: Nome e/ou ID do grupo. 
	\end{itemize}


\newpage
\section{Testar a exibição de informações de um item cadastrado no sistema}

	\item Caso de Teste - SIGP-CT-015
	\begin{itemize}
	\item Descrição: Exibir principais informações de uma \textbf{linha de pesquisa}.
	\item Pré-condições: Ser usuário. Ter uma \emph{View} com um \emph{link} para outra \emph{View} em que seja exibida a \textbf{descrição de uma linha de pesquisa}, e listagens das \textbf{publicações}, \textbf{filiacoes} e \textbf{projetos} associados à linha de pesquisa em questão.
	\item Pós-condições: Exibir na \emph{View}, a \textbf{descrição da linha de pesquisa}, uma lista de \textbf{filiacoes do grupo}, lista de \textbf{publicações da linha de pesquisa} e lista de \textbf{projetos} associados à linha de pesquisa.
	\item Dados necessários: Nome e/ou ID da linha de pesquisa.
	\end{itemize}

	\item Caso de Teste - SIGP-CT-016
	\begin{itemize}
	\item Descrição: Exibir principais informações de um \textbf{projeto}.
	\item Pré-condições: Ser usuário. Ter uma \emph{View} para exibir principais informações de um projeto.
	\item Pós-condições: Exibir a \textbf{descrição do projeto}, uma lista de \textbf{filiações do projeto} e uma lista de \textbf{publicações do projeto} na \emph{View}.
	\item Dados necessários: Nome e/ou ID do projeto.
	\end{itemize}

\newpage
\section{Testar a capacidade de reorganização da exibição de dados de um item em uma \emph{View}}

	\item Caso de Teste - SIGP-CT-017
	\begin{itemize}
	\item Descrição: Ordenar \textbf{publicações} por tipo, ano ou veículo.
	\item Pré-condições: Ser usuário. Ter uma \emph{View} para listar publicações de um grupo.
	\item Pós-condições: Uma \emph{View} com três controles (botões/links) para ordenar uma llista de \textbf{publicações de um grupo} por tipo, ano ou veículo.
	\item Dados necessários: Nome e/ou ID de um grupo, projeto ou linha de pesquisa.
	\end{itemize}

\newpage
\section{Testar a tentativa de cadastrar ítens já cadastrados no sistema.}

	\item Caso de Teste - SIGP-CT-018
	\begin{itemize}
	\item Descrição: Tentar cadastrar um \textbf{usuário} já cadastrado no sistema.
	\item Pré-condições: Ser administrador. Ter uma \emph{View} com formulário para cadastro de usuário. Banco de Dados criado.
	\item Pós-condições: Exibir uma \emph{View} com uma mensagem de erro, informando que este registro já existe no BD.
	\item Dados necessários: Login, Senha, Tipo, Avatar
	\end{itemize}


\item  Caso de Teste - SIGP-CT-019
	\begin{itemize}
	\item Descrição: Tentar cadastrar um \textbf{contribuinte} já cadastrado no sistema.
	\item Pré-condições: Ser administrador. Ter uma \emph{View} com formulário para cadastro de contribuinte com entradas para especificar contribuinte que é usuário. Banco de Dados criado.
	\item Pós-condições: Exibir uma \emph{View} com uma mensagem de erro, informando que este registro já existe no BD.
	\item Dados necessários: Nome, Publicações, Filiação, Login, Senha, Avatar, Tipo
	\end{itemize}

	\item Caso de Teste - SIGP-CT-020
	\begin{itemize}
	\item Descrição: Tentar cadastrar um \textbf{grupo} já cadastrado no sistema.
	\item Pré-condições: Ser adminstrador. Ter uma \emph{View} com formulário para cadastro de grupo. Ter o Banco de Dados criado.
	\item Pós-condições: Exibir uma \emph{View} com uma mensagem de erro, informando que este registro já existe no BD.
	\item Dados necessários: Nome, Subgrupo, Lista de filiações, Lista de disciplinas, Lista de linhas de pesquisa.
	\end{itemize}

	\item Caso de Teste - SIGP-CT-021
	\begin{itemize}
	\item Descrição: Tentar cadastrar um \textbf{subgrupo} já cadastrado no sistema.
	\item Pré-condições: Ser administrador. Ter uma \emph{View} com formulário para cadastro de subgrupo de um grupo.Ter o Banco de Dados criado.
	\item Pós-condições: Exibir uma \emph{View} com uma mensagem de erro, informando que este registro já existe no BD.
	\item Dados necessários: Nome, Grupo, Lista de filiações, Lista de disciplinas, Lista de linhas de pesquisa.
	\end{itemize}

\newpage

	\item Caso de Teste - SIGP-CT-022
	\begin{itemize}
	\item Descrição: Tentar cadastrar uma \textbf{publicação} já cadastrada no sistema.
	\item Pré-condições: Ser usuário. Ter uma \emph{View} com formulário para cadastro de publicação.Ter o Banco de Dados criado.
	\item Pós-condições: Exibir uma \emph{View} com uma mensagem de erro, informando que este registro já existe no BD.
	\item Dados necessários: Titulo, Veiculo, Data, Lista de contribuintes, lista de projetos
	\end{itemize}

	\item Caso de Teste - SIGP-CT-023
	\begin{itemize}
	\item Descrição: Tentar cadastrar uma \textbf{linha de pesquisa} já cadastrada no sistema.
	\item Pré-condições: Ser usuário. Ter uma \emph{View} com formulário para cadastrar linha de pesquisa. Banco de Dados criado.
		\item Pós-condições: Exibir uma \emph{View} com uma mensagem de erro, informando que este registro já existe no BD.
	\item Dados necessários: Nome. Lista de projetos na linha de pesquisa. Lissta de grupos.
	\end{itemize}

	\item Caso de Teste - SIGP-CT-024
	\begin{itemize}
	\item Descrição: Tentar cadastrar um \textbf{projeto} já cadastrado no sistema.
	\item Pré-condições: Banco de Dados criado.
	\item Pós-condições: Exibir uma \emph{View} com uma mensagem de erro, informando que este registro já existe no BD.
	\item Dados necessários: Nome do projeto, Descrição do projeto, Nome da Agência de Fomento financiadora, Linha de Pesquisa, Lista de filiações, Lista de publicações.
	\end{itemize}

\newpage
\section{Testar a tentativa de consultar um item não-cadastrado no sistema.}

	\item Caso de Teste - SIGP-CT-025
	\begin{itemize}
	\item Descrição: Tentar consultar um item não-cadastrado no sistema.
	\item Pré-condições: Banco de Dados criado.
	\item Pós-condições: Exibir uma \emph{View} com uma mensagem de erro, informando que este registro não existe no BD.
	\item Dados necessários: Nome do item.
	\end{itemize}

\newpage
\section{Testar a eliminação de um item cadastrado no sistema uma vez.}
\label{sec:elimumavez}

	\item Caso de Teste - SIGP-CT-026
	\begin{itemize}
	\item Descrição: Eliminar uma vez um contribuinte cadastrado no sistema.
	\item Pré-condições: Ser administrador. Ter uma \emph{View} para exibição do registro de um contribuinte, com um controle (botão) para eliminar este registro. Banco de Dados criado.
	\item Pós-condições: Um registro de contribuinte a menos no BD e apresentação na listagem subsequente.
	\item Dados necessários: Nome ou ID do contribuinte.
	\end{itemize}

	\item Caso de Teste - SIGP-CT-027
	\begin{itemize}
	\item Descrição: Eliminar uma vez um usuário cadastrado no sistema.
	\item Pré-condições: Ser administrador. Ter uma \emph{View} para exibição do registro de um usuário, com um controle (botão) para eliminar este registro. Banco de Dados criado.
	\item Pós-condições: Um registro de usuário a menos no BD e apresentação na listagem subsequente.
	\item Dados necessários: Nome ou ID do usuário.
	\end{itemize}

          \item Caso de Teste - SIGP-CT-028
	\begin{itemize}
	\item Descrição: Eliminar uma vez um grupo cadastrado no sistema.
	\item Pré-condições: Ser administrador. Ter uma \emph{View} para exibição do registro de um grupo, com um controle (botão) para eliminar este registro. Banco de Dados criado.
	\item Pós-condições: Um registro de grupo a menos no BD e apresentação na listagem subsequente.
	\item Dados necessários: Nome ou ID do grupo.
	\end{itemize}

	\item Caso de Teste - SIGP-CT-029
	\begin{itemize}
	\item Descrição: Eliminar uma vez um subgrupo cadastrado no sistema.
	\item Pré-condições: Ser administrador. Ter uma \emph{View} para exibição do registro de um subgrupo, com um controle (botão) para eliminar este registro. Banco de Dados criado.
	\item Pós-condições: Um registro de subgrupo a menos no BD e apresentação na listagem subsequente.
	\item Dados necessários: Nome ou ID do subgrupo.
	\end{itemize}

\newpage

	\item Caso de Teste - SIGP-CT-030
	\begin{itemize}
	\item Descrição: Eliminar uma vez uma publicação cadastrada no sistema.
	\item Pré-condições: Ser administrador. Ter uma \emph{View} para exibição do registro de uma publicação, com um controle (botão) para eliminar este registro. Banco de Dados criado.
	\item Pós-condições: Um registro de publicação a menos no BD e apresentação na listagem subsequente.
	\item Dados necessários: Título e ou ID da publicação.
	\end{itemize}


	\item Caso de Teste - SIGP-CT-031
	\begin{itemize}
	\item Descrição: Eliminar uma vez uma linha de pesquisa cadastrada no sistema.
	\item Pré-condições: Ser administrador. Ter uma \emph{View} para exibição do registro de uma linha de pesquisa, com um controle (botão) para eliminar este registro. Banco de Dados criado.
	\item Pós-condições: Um registro de linha de pesquisa a menos no BD e apresentação na listagem subsequente.
	\item Dados necessários: Nome e ou ID da linha de pesquisa.
	\end{itemize}

	\item Caso de Teste - SIGP-CT-032
	\begin{itemize}
	\item Descrição: Eliminar uma vez um projeto cadastrado no sistema.
	\item Pré-condições: Ser administrador. Ter uma \emph{View} para exibição do registro de um projeto, com um controle (botão) para eliminar este registro. Banco de Dados criado.
	\item Pós-condições: Um registro de projeto a menos no BD e apresentação na listagem subsequente.
	\item Dados necessários: Título e ou ID da publicação.
	\end{itemize}

\newpage
\section{Testar a eliminação de um item cadastrado no sistema duas vezes.}

Repetir os passos da seção~\ref{sec:elimumavez} para cada item. Exemplo:

	\item Caso de Teste - SIGP-CT-033
	\begin{itemize}
	\item Descrição: Tentar eliminar duas vezes uma pessoa cadastrada no sistema.
	\item Pré-condições: Ser administrador. Ter uma \emph{View} para exibição do registro de um contribuinte, com um controle (botão) para eliminar este registro. Banco de Dados criado.
	\item Pós-condições: Exibir uma \emph{View} com uma mensagem de erro, informando que este registro não pode ser elimiado, pois não existe no BD.
	\item Dados necessários: Nome ou ID do contribuinte
	\end{itemize}

\newpage
\section{Testar a atualização de cada campo de um item cadastrado no sistema.}

	\item Caso de Teste - SIGP-CT-034
	\begin{itemize}
	\item Descrição: Tentar atualizar duas vezes cada campo de um \textbf{contribuinte} cadastrado no sistema.
	\item Pré-condições: Ser administrador. Ter uma \emph{View} para exibição do registro de um contribuinte, com campos editáveis e um botão de salvar alterações. Banco de Dados criado.
	\item Pós-condições: Verificar que a segunda versão está armazenada por meio da \emph{View} atualizada. BD atualizado.
	\item Dados necessários: Nome, Lista de publicações, Filiação, Usuário
	\end{itemize}

\newpage
\section{Testar a modificação de um item cadastrado no sistema.}
	\item Caso de Teste - SIGP-CT-035
	\begin{itemize}
	\item Descrição: Tentar modificar o registro de um \textbf{usuário} cadastrado no sistema.
	\item Pré-condições: Ser administrador. Ter uma \emph{View} para exibição do registro de um usuário, com campos editáveis e um botão de salvar alterações. Banco de Dados criado.
	\item Pós-condições: Verificar que a alteração está armazenada por meio da \emph{View} atualizada. BD atualizado.
	\item Dados necessários: Login, Senha, Tipo, Avatar
	\end{itemize}


	\item  Caso de Teste - SIGP-CT-036
	\begin{itemize}
	\item Descrição: Tentar modificar o registro de uma \textbf{filiação} cadastrada no sistema.
	\item Pré-condições: Ser administrador. Ter uma \emph{View} para exibição do registro de uma filiação, com campos editáveis e um botão de salvar alterações. Banco de Dados criado.
	\item Pós-condições:  Verificar que a alteração está armazenada por meio da \emph{View} atualizada. BD atualizado.
	\item Dados necessários: Contribuinte, Lista de Grupos, Lista de Projetos
	\end{itemize}



	\item Caso de Teste - SIGP-CT-037
	\begin{itemize}
	\item Descrição: Tentar modificar o registro de um  \textbf{grupo} cadastrado no sistema.
	\item Pré-condições: Ser administrador. Ter uma \emph{View} para exibição do registro de um grupo, com campos editáveis e um botão de salvar alterações. Banco de Dados criado.
\item Pós-condições:  Verificar que a alteração está armazenada por meio da \emph{View} atualizada. BD atualizado.
	\item Dados necessários: Nome, Subgrupo, Lista de linhas de pesquisa, Lista de filiações.
	\end{itemize}

	\item Caso de Teste - SIGP-CT-038
	\begin{itemize}
	\item Descrição: Tentar modificar o registro de um  \textbf{subgrupo} cadastrado no sistema.
	\item Pré-condições: Ser administrador. Ter uma \emph{View} para exibição do registro de um subgrupo, com campos editáveis e um botão de salvar alterações. Banco de Dados criado.
	\item Pós-condições:  Verificar que a alteração está armazenada por meio da \emph{View} atualizada. BD atualizado.
	\item Dados necessários:  Nome, Grupo, Lista de linhas de pesquisa, Lista de filiações.
	\end{itemize}

\newpage

	\item Caso de Teste - SIGP-CT-039
	\begin{itemize}
	\item Descrição: Tentar modificar o registro de uma \textbf{publicação} cadastrada no sistema.
\item Pré-condições: Ser administrador. Ter uma \emph{View} para exibição do registro de uma publicação, com campos editáveis e um botão de salvar alterações. Banco de Dados criado.
	\item Pós-condições:  Verificar que a alteração está armazenada por meio da \emph{View} atualizada. BD atualizado.
	\item Dados necessários: Título, veículo, data, Lista de contribuintes, Lista de projetos.
	\end{itemize}

	\item Caso de Teste - SIGP-CT-040
	\begin{itemize}
	\item Descrição: Tentar modificar o registro de uma \textbf{linha de pesquisa} cadastrada no sistema.
	\item Pré-condições: Ser administrador. Ter uma \emph{View} para exibição do registro de uma linha de pesquisa, com campos editáveis e um botão de salvar alterações. Banco de Dados criado.
	\item Pós-condições:  Verificar que a alteração está armazenada por meio da \emph{View} atualizada. BD atualizado.
	\item Dados necessários: Nome. Lista de Grupos. Lista de projetos na linha de pesquisa.
	\end{itemize}

	\item Caso de Teste - SIGP-CT-041
	\begin{itemize}
	\item Descrição:  Tentar modificar o registro de uma \textbf{projeto} cadastrada no sistema.
\item Pré-condições: Ser administrador. Ter uma \emph{View} para exibição do registro de um projeto, com campos editáveis e um botão de salvar alterações. Banco de Dados criado.
	\item Pós-condições: O registro de mais um projeto no BD e na listagem subsequente.
	\item Dados necessários: Nome do projeto, Descrição do projeto, Nome da Agência de Fomento financiadora, lista de filiações, lista de publicações.
	\end{itemize}


\newpage
\section{Testar a impressão de uma \emph{View}}
	\item Caso de Teste - SIGP-CT-042
	\begin{itemize}
	\item Descrição: Imprimir uma \emph{View} do sistema.
	\item Pré-condições: Ser membro-usuário. Ter uma \emph{View} do sistema exibida na tela do computador com um ícone ou botão para impressão de uma versão amigável.
	\item Pós-condições: Impressão de uma versão amigável.
	\item Dados necessários: Nenhum dado.
	\end{itemize}





\end{itemize}

\end{document}
